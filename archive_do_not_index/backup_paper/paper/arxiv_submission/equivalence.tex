\section{Equivalence of Characterizations}\label{sec:equivalence}

In this section, we establish the formal equivalence between the \HAPD{} algorithm approach and the matrix-based characterization of cubic irrationals. This equivalence demonstrates that our solution to Hermite's problem is robust and theoretically well-founded, with multiple complementary perspectives supporting the same conclusion.

\subsection{Structural Equivalence}

We begin by proving that the structures underlying both approaches are fundamentally the same.

\begin{theorem}[Structural Equivalence]\label{thm:structural_equivalence}
Let $\alpha$ be a real number. The following statements are equivalent:
\begin{enumerate}
    \item $\alpha$ is a cubic irrational.
    \item The sequence produced by the \HAPD{} algorithm with input $\alpha$ is eventually periodic.
    \item There exists a $3 \times 3$ companion matrix $C$ with rational entries such that the characteristic polynomial of $C$ is irreducible over $\Q$ and $\tr(C^k) = \alpha^k + \beta^k + \gamma^k$ for all $k \geq 1$, where $\beta$ and $\gamma$ are the other roots of the minimal polynomial of $\alpha$.
\end{enumerate}
\end{theorem}

\begin{proof}
$(1) \Rightarrow (2)$: This is Theorem \ref{thm:cubic_periodic}.

$(2) \Rightarrow (1)$: This is Theorem \ref{thm:only_cubic_periodic}.

$(1) \Rightarrow (3)$: This is the forward direction of Theorem \ref{thm:matrix_cubic}.

$(3) \Rightarrow (1)$: This is the reverse direction of Theorem \ref{thm:matrix_cubic}.

Since all implications hold, the three statements are equivalent.
\end{proof}

\subsection{Algebraic Connection}

We now establish a deeper algebraic connection between the \HAPD{} algorithm and the matrix approach, showing how the algorithm's operations relate to the matrix properties.

\begin{theorem}[Algebraic Connection]\label{thm:algebraic_connection}
If $\alpha$ is a cubic irrational with minimal polynomial $f(x) = x^3 + px^2 + qx + r$, then:
\begin{enumerate}
    \item The periodicity of the \HAPD{} algorithm corresponds to the action of a specific finitely generated subgroup of $\GL(3, \Q)$ on projective space.
    \item This subgroup is related to the unit group of the ring of integers in the cubic field $\Q(\alpha)$.
    \item The traces of powers of the companion matrix $C_f$ encode the same information as the periodic pattern in the \HAPD{} algorithm.
\end{enumerate}
\end{theorem}

\begin{proof}
\begin{enumerate}
    \item From Proposition \ref{prop:matrix_hapd}, each iteration of the \HAPD{} algorithm corresponds to applying a transformation matrix to the current state. The sequence of these matrices generates a subgroup of $\GL(3, \Q)$ that acts on the projective space. By Theorem \ref{thm:matrix_periodicity}, periodicity occurs when a product of these matrices maps the initial point to a scalar multiple of itself.
    
    \item The unit group of the ring of integers in $\Q(\alpha)$ acts on the field, and this action can be represented in terms of matrices acting on the standard basis $\{1, \alpha, \alpha^2\}$. The \HAPD{} algorithm effectively captures a discrete subset of this action, related to the fundamental units of the cubic field.
    
    \item The periodic pattern in the \HAPD{} algorithm provides a sequence of integer pairs that encode how the projective point evolves. The traces of powers of the companion matrix, on the other hand, provide the power sums of the roots. Both encode the minimal polynomial of $\alpha$, just in different ways: the \HAPD{} algorithm through its dynamic behavior, and the trace formula through direct algebraic relations.
\end{enumerate}
\end{proof}

\begin{corollary}[Information Content]\label{cor:information_content}
Both approaches (\HAPD{} algorithm and matrix traces) contain sufficient information to uniquely determine the cubic field $\Q(\alpha)$ up to isomorphism.
\end{corollary}

\begin{proof}
From a cubic irrational $\alpha$, both approaches can be used to determine the coefficients of its minimal polynomial, which fully characterizes the field extension $\Q(\alpha)$ up to isomorphism.
\end{proof}

\subsection{Computational Perspective}

We next examine the equivalence from a computational perspective, showing that both approaches lead to practical algorithms with comparable properties.

\begin{theorem}[Computational Equivalence]\label{thm:computational_equivalence}
The following computational procedures are equivalent in their ability to detect cubic irrationals:
\begin{enumerate}
    \item Running the \HAPD{} algorithm and detecting periodicity in the output sequence.
    \item Finding a candidate minimal polynomial of degree 3 and verifying that its companion matrix $C$ satisfies $\tr(C^k) \approx \alpha^k + \beta^k + \gamma^k$ for several values of $k$.
\end{enumerate}
\end{theorem}

\begin{proof}
Both procedures correctly identify a real number as a cubic irrational if and only if it actually is one, as established by Theorems \ref{thm:cubic_periodic}, \ref{thm:only_cubic_periodic}, and \ref{thm:matrix_cubic}.

From a computational perspective, both approaches involve similar operations:
\begin{itemize}
    \item The \HAPD{} algorithm applies a sequence of transformations and checks for repetition in projective space.
    \item The matrix approach computes powers of a matrix and checks trace relations.
\end{itemize}

The key difference is in the specific computations performed, but both methods effectively detect the same underlying mathematical property: whether $\alpha$ generates a cubic field extension over $\Q$.
\end{proof}

\begin{proposition}[Complexity Comparison]\label{prop:complexity_comparison}
For a cubic irrational $\alpha$ with minimal polynomial having coefficients bounded by $M$:
\begin{enumerate}
    \item The \HAPD{} algorithm requires $O(M^3)$ iterations to detect periodicity, with each iteration performing $O(1)$ arithmetic operations.
    \item The matrix approach requires computing $O(1)$ powers of a $3 \times 3$ matrix and checking trace relations, with each matrix multiplication requiring $O(1)$ arithmetic operations.
\end{enumerate}
\end{proposition}

\begin{proof}
For the \HAPD{} algorithm, the number of iterations required to detect periodicity is bounded by the size of the fundamental domain of the Dirichlet group, which scales with the discriminant, yielding the $O(M^3)$ bound as established in Proposition \ref{prop:complexity}.

For the matrix approach, a fixed number of trace checks (typically 3-4) is sufficient to verify with high confidence that $\alpha$ is a cubic irrational, once a candidate minimal polynomial is found. Each trace check involves computing the $k$-th power of the companion matrix, which requires $O(1)$ matrix multiplications using exponentiation by squaring.
\end{proof}

\begin{remark}
While the matrix approach may appear more efficient in terms of asymptotic complexity, the \HAPD{} algorithm has the advantage of working directly with the real number $\alpha$ without requiring prior knowledge of its minimal polynomial. The matrix approach requires first finding a candidate minimal polynomial, which itself can be computationally intensive.
\end{remark}

\subsection{Theoretical Unification}

We now present a unified theoretical framework that encompasses both approaches, showing how they relate to the broader context of algebraic number theory and geometric structures.

\begin{theorem}[Theoretical Unification]\label{thm:theoretical_unification}
Let $\alpha$ be a cubic irrational. The following mathematical structures are all equivalent characterizations of $\alpha$:
\begin{enumerate}
    \item The cubic field extension $\Q(\alpha)/\Q$ with its associated Galois action.
    \item The periodic dynamics of the \HAPD{} algorithm in projective space.
    \item The spectrum and trace properties of the companion matrix of the minimal polynomial of $\alpha$.
    \item The action of the Dirichlet group $\Gamma_{\Q(\alpha)}$ on projective space with its fundamental domain.
\end{enumerate}
\end{theorem}

\begin{proof}
The equivalence of these characterizations follows from the combined results of Sections \ref{sec:galois_theory}, \ref{sec:hapd_algorithm}, and \ref{sec:matrix_approach}.

The cubic field extension $\Q(\alpha)/\Q$ fundamentally determines all algebraic properties of $\alpha$. The Galois action on the roots of the minimal polynomial corresponds to the spectrum of the companion matrix, and the trace properties of powers of this matrix encode the power sums of these roots.

The \HAPD{} algorithm captures the discrete action of a specific subgroup related to the cubic field structure, and its periodicity is a manifestation of the finiteness of the fundamental domain of the associated Dirichlet group in projective space.

All of these perspectives are different ways of viewing the same underlying mathematical structure: the cubic field $\Q(\alpha)$ and its intrinsic properties.
\end{proof}

\begin{corollary}[Completeness of Solution]\label{cor:completeness}
Our characterization of cubic irrationals through either the \HAPD{} algorithm or the matrix approach provides a complete solution to Hermite's problem, in the sense that it correctly identifies all cubic irrationals and only cubic irrationals.
\end{corollary}

\begin{proof}
This follows directly from Theorem \ref{thm:structural_equivalence}.
\end{proof}

\begin{remark}
While our solution differs from what Hermite might have initially envisioned—a direct analogue of continued fractions in one-dimensional space—we have shown in Section \ref{sec:galois_theory} that such a direct analogue cannot exist. Our solution using the \HAPD{} algorithm in three-dimensional projective space is the natural generalization, achieving Hermite's goal in a more sophisticated context.
\end{remark}

\subsection{Generalizations and Extensions}

Finally, we discuss possible generalizations of our approach to algebraic numbers of higher degree, providing a roadmap for extending the solution to Hermite's problem beyond the cubic case.

\begin{theorem}[Generalization to Higher Degrees]\label{thm:generalization}
The principles underlying both the \HAPD{} algorithm and the matrix approach can be extended to characterize algebraic irrationals of degree $n > 3$, with the following modifications:
\begin{enumerate}
    \item The \HAPD{} algorithm generalizes to work with $n$-dimensional projective space, initialized with the tuple $(\alpha, \alpha^2, \ldots, \alpha^{n-1}, 1)$.
    \item The matrix approach generalizes to using $n \times n$ companion matrices and checking trace relations involving the sum of $k$-th powers of all $n$ roots.
\end{enumerate}
\end{theorem}

\begin{proof}
The generalization follows the same principles as the cubic case:
\begin{itemize}
    \item For an algebraic irrational of degree $n$, the field extension $\Q(\alpha)/\Q$ has degree $n$, with basis $\{1, \alpha, \alpha^2, \ldots, \alpha^{n-1}\}$.
    \item The companion matrix of the minimal polynomial has size $n \times n$ and encodes the same algebraic relations.
    \item The projective space increases to dimension $n-1$, but the principle of detecting periodicity through the finiteness of fundamental domains of appropriate discrete groups remains valid.
\end{itemize}

The detailed proof would follow the structure of our cubic case, with appropriate modifications for the higher-dimensional setting.
\end{proof}

\begin{remark}
While the theoretical generalization is straightforward, the practical implementation becomes increasingly complex for higher degrees, due to the growth in dimensionality and the need for more sophisticated methods to detect periodicity in higher-dimensional projective spaces.
\end{remark}

\begin{proposition}[Generalized Hermite Problem]\label{prop:generalized_hermite}
For each positive integer $n$, there exists an algorithm that, for any real number $\alpha$, produces a sequence that is eventually periodic if and only if $\alpha$ is an algebraic irrational of degree exactly $n$.
\end{proposition}

\begin{proof}
This follows from the generalization outlined in Theorem \ref{thm:generalization}, combined with the theoretical framework established in this paper. The detailed construction for each $n$ would require adapting the \HAPD{} algorithm to the appropriate dimensionality and proving the analogous periodicity properties.
\end{proof}

\begin{remark}
The existence of such generalized algorithms completes the pattern that Hermite sought to extend: just as periodic decimal expansions characterize rational numbers, and periodic continued fractions characterize quadratic irrationals, there exist $n$-dimensional generalizations that characterize algebraic irrationals of degree $n$ through periodicity.
\end{remark}

This establishes the equivalence of our approaches and places them within a broader theoretical context, demonstrating the robustness and completeness of our solution to Hermite's problem.
