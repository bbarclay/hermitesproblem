\section{Conclusion}\label{sec:conclusion}

We solved Hermite's problem via two approaches: HAPD and modified sin²-algorithm. Both characterize cubic irrationals through periodicity. Key contributions include:

\begin{itemize}
\item Projective space approach handling cubic irrationals with complex conjugate roots
\item Mathematical proof of periodicity for all cubic irrationals
\item Numerical validation confirming theoretical predictions
\item Efficient implementations with polynomial-time complexity
\end{itemize}

HAPD achieves Hermite's goal using projective geometry, while the modified sin²-algorithm extends Karpenkov's work to all cubic irrationals.

Extensions to higher algebraic degrees offer promising research directions. These approaches could characterize algebraic irrationals of arbitrary degree through periodicity in appropriate representation spaces.

Natural generalizations include:
\begin{enumerate}
\item Higher-degree algebraic irrationals
\item Optimized implementations with quantified complexity bounds
\item Geometric interpretations in projective and hyperbolic spaces
\item Applications to integer relation detection and lattice reduction
\item Connections to ergodic theory via Dirichlet groups
\end{enumerate}
