\section{Conclusion}\label{sec:conclusion}

We have solved Hermite's problem through two approaches: the \HAPD algorithm and the modified sin²-algorithm. Both characterize cubic irrationals with complex conjugate roots through periodicity. Our solution consists of:

\begin{enumerate}
\item Proof that cubic irrationals cannot have periodic continued fraction expansions, necessitating higher-dimensional methods
\item The \HAPD algorithm operating in projective space, producing eventually periodic sequences precisely for cubic irrationals
\item A modified sin²-algorithm with phase-preserving floor function handling complex conjugates
\item Proof of equivalence between both approaches, demonstrating periodicity as an intrinsic property of cubic fields
\end{enumerate}

Numerical validation confirms both algorithms correctly classify number types across test cases including different Galois group structures and discriminant edge cases.

This approach generalizes naturally to several directions:
\begin{enumerate}
\item Higher-degree algebraic irrationals
\item Optimized implementations with quantified complexity bounds
\item Geometric interpretations in projective and hyperbolic spaces
\item Applications to integer relation detection and lattice basis reduction
\item Connections to ergodic theory via Dirichlet groups on homogeneous spaces
\end{enumerate}
