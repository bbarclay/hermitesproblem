\section{Conclusion}\label{sec:conclusion}

We have presented three novel approaches to Hermite's classical problem of characterizing cubic irrationals through periodicity. Our unified solution bridges algebraic number theory, projective geometry, and computational mathematics.

The \HAPD{} algorithm provides a geometric characterization operating in projective space $\mathbb{P}^2(\mathbb{R})$, generating periodic sequences if and only if the input is a cubic irrational. The matrix approach offers an algebraic perspective through companion matrices and trace sequences, mathematically equivalent to the \HAPD{} algorithm but with computational advantages. The subtractive algorithm provides a numerically stable variant that preserves the theoretical properties while enhancing practical implementation.

Our three key contributions are:
\begin{enumerate}
\item \textbf{Complete Characterization:} We provide necessary and sufficient conditions for cubic irrationality through periodicity, addressing Hermite's problem comprehensively.
\item \textbf{Multiple Perspectives:} Our three equivalent approaches offer insights from geometric, algebraic, and computational viewpoints, enhancing understanding of the underlying mathematical structures.
\item \textbf{Practical Implementation:} The algorithms are accompanied by detailed analysis of computational complexity and numerical considerations, facilitating practical applications.
\end{enumerate}

The theoretical analysis is complemented by numerical validation confirming the efficacy of our approaches. The algorithms correctly identify cubic irrationals with high accuracy, distinguishing them from other number types.

Our solution to Hermite's problem extends the classical theory of continued fractions to cubic irrationals, establishing a fundamental connection between algebraic degree and periodicity that parallels Lagrange's theorem for quadratic irrationals.

\subsection{Future Work}

Building on the foundations established in this paper, several promising directions for future research emerge:

\begin{enumerate}
\item \textbf{Higher Degree Generalization:} A natural extension of our work is to algebraic numbers of degree greater than three. We conjecture:

\begin{conjecture}[Higher Degree Generalization]\label{conj:higher_degree}
For any integer $n \geq 2$, there exists an algorithm operating in $n$-dimensional projective space that produces eventually periodic sequences if and only if the input is an algebraic number of degree $n$.
\end{conjecture}

The key components required for such a generalization include:
\begin{itemize}
\item A representation in $n$-dimensional projective space that captures the algebraic structure of degree-$n$ fields
\item A transformation that preserves the field structure while allowing for efficient encoding of the transformation parameters
\item A periodicity detection mechanism that can identify equivalence classes in the projective space
\end{itemize}

\item \textbf{Computational Optimizations:} Develop specialized data structures and algorithms to improve the practical efficiency of periodicity detection, particularly for high-degree cases.

\item \textbf{Applications in Number Theory:} Investigate applications to other number-theoretic problems, such as Diophantine approximation, irrationality measures, and transcendence proofs.

\item \textbf{Quantum Computing Implementation:} Explore quantum algorithms that could potentially offer polynomial speedup for periodicity detection in algebraic numbers.

\item \textbf{Connection to Ergodic Theory:} Further develop the relationship between our algorithms and ergodic theory, particularly the dynamics on homogeneous spaces.
\end{enumerate}

These directions represent exciting possibilities for extending the mathematical and computational framework developed in this paper, potentially yielding insights into both fundamental number theory and practical algorithmic applications.

\subsection{Interactive Materials}

To facilitate deeper understanding and exploration of the algorithms presented in this paper, we have developed interactive visualizations and computational tools that are freely available online. These resources allow readers to:

\begin{itemize}
\item Experiment with the HAPD algorithm and observe its periodicity detection in real-time
\item Visualize projective space transformations and their relationship to cubic irrationals
\item Test the matrix verification approach with custom inputs
\item Explore the subtractive algorithm's behavior on various cubic polynomials
\item Compare the performance characteristics of all three approaches
\end{itemize}

These interactive materials, along with source code and additional examples, can be accessed at \url{https://bbarclay.github.io/hermitesproblem/}. We encourage interested readers to use these tools to develop intuition about the theoretical concepts and to explore the algorithms' behavior with custom inputs beyond the examples presented in this paper.
