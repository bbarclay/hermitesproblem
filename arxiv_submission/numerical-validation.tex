\section{Numerical Validation and Implementation}\label{sec:numerical_validation}

In this section, we provide numerical validation of our theoretical results through concrete implementations of both the \HAPD{} algorithm and the matrix-based approach. We present empirical evidence confirming that our methods correctly distinguish cubic irrationals from other number types and analyze the practical challenges of implementation.

\subsection{Implementation of the HAPD Algorithm}

We begin with a detailed implementation of the \HAPD{} algorithm, addressing precision requirements and numerical stability considerations.

\begin{figure}[htbp]
\begin{minipage}{\textwidth}
\centering
\includegraphics[width=\textwidth]{figures/output/cubic_trajectories.pdf}
\caption{Trajectories of cubic irrationals under the HAPD algorithm iterations, showing the periodic patterns that emerge for different cubic irrationals with complex conjugate roots. Each color represents the path of a distinct cubic number.}
\label{fig:cubic_trajectories}
\end{minipage}
\end{figure}

\begin{algorithm}
\caption{Implementation of the \HAPD{} Algorithm}
\label{alg:hapd_implementation}
\begin{algorithmic}[1]
\Procedure{HAPD}{$\alpha$, max\_iterations, tolerance}
    \State $v_1 \gets \alpha$
    \State $v_2 \gets \alpha^2$
    \State $v_3 \gets 1$
    \State triples $\gets$ empty list
    \State pairs $\gets$ empty list
    
    \For{$i \gets 1$ to max\_iterations}
        \State $a_1 \gets \lfloor v_1/v_3 \rfloor$
        \State $a_2 \gets \lfloor v_2/v_3 \rfloor$
        \State $r_1 \gets v_1 - a_1 \cdot v_3$
        \State $r_2 \gets v_2 - a_2 \cdot v_3$
        \State $v_3^{\text{new}} \gets v_3 - a_1 \cdot r_1 - a_2 \cdot r_2$
        
        \State pairs.append($(a_1, a_2)$)
        
        \If{$|v_3^{\text{new}}| < \text{tolerance}$}
            \State \Return pairs, "Terminated (likely rational)"
        \EndIf
        
        \State $v_1 \gets r_1$
        \State $v_2 \gets r_2$
        \State $v_3 \gets v_3^{\text{new}}$
        
        \State $\text{triple} \gets (v_1, v_2, v_3)$
        \State Normalize triple to have norm 1
        
        \For{$j \gets 0$ to $\text{triples.length} - 1$}
            \If{$\text{ProjectivelyEquivalent(triple, triples[j], tolerance)}$}
                \State \Return pairs, "Periodic with preperiod $j$ and period $i-j$"
            \EndIf
        \EndFor
        
        \State triples.append(triple)
    \EndFor
    
    \State \Return pairs, "No periodicity detected within max\_iterations"
\EndProcedure

\Function{ProjectivelyEquivalent}{triple1, triple2, tolerance}
    \State Normalize both triples to have norm 1
    \State $\text{dotProduct} \gets \sum_{i=1}^3 \text{triple1}[i] \cdot \text{triple2}[i]$
    \State \Return $||\text{dotProduct}| - 1| < \text{tolerance}$
\EndFunction
\end{algorithmic}
\end{algorithm}

\begin{remark}
The algorithm includes normalization of each triple to unit length to improve numerical stability when comparing projective points. The function \textsc{ProjectivelyEquivalent} checks if two normalized triples represent the same point in projective space, allowing for a small numerical tolerance.
\end{remark}

\begin{proposition}[Numerical Precision Requirements]\label{prop:numerical_precision}
For reliable detection of periodicity in the \HAPD{} algorithm for a cubic irrational with minimal polynomial coefficients bounded by $M$:
\begin{enumerate}
    \item Floating-point precision of at least $O(\log M)$ bits is required
    \item The comparison tolerance should be set to approximately $2^{-p/2}$, where $p$ is the number of bits of precision
\end{enumerate}
\end{proposition}

\begin{proof}
The algorithm involves computing ratios and remainders in each iteration. For a cubic irrational with coefficients bounded by $M$, the entries in the transformation matrices are also bounded by polynomials in $M$.

Over the course of $O(M^3)$ iterations needed to detect periodicity, numerical errors can accumulate, potentially leading to false positives or negatives in periodicity detection. With $p$ bits of precision, the maximum attainable accuracy is approximately $2^{-p}$.

When comparing projective points, we compute the dot product of normalized vectors, which should be exactly 1 for identical points or -1 for antipodal points. Allowing for numerical errors, the tolerance should be on the order of $2^{-p/2}$ to account for error accumulation while still distinguishing truly distinct points.
\end{proof}

\subsection{Test Cases and Results}

We now present results from applying the \HAPD{} algorithm to various types of numbers, demonstrating its effectiveness in identifying cubic irrationals.

\begin{table}[h]
\centering
\caption{Results of the \HAPD{} Algorithm for Different Number Types}
\label{tab:hapd_results}
\begin{tabular}{|p{2.2cm}|p{2.8cm}|p{2.7cm}|p{1.8cm}|p{1.5cm}|}
\hline
\textbf{Number} & \textbf{Type} & \textbf{Behavior} & \textbf{Pre-\newline period} & \textbf{Period} \\
\hline
$\sqrt{2}$ & Quadratic Irr. & Non-periodic & - & - \\
$\sqrt{3}$ & Quadratic Irr. & Non-periodic & - & - \\
$\frac{1+\sqrt{5}}{2}$ & Quadratic Irr. & Non-periodic & - & - \\
\hline
$2^{1/3}$ & Cubic Irr. & Periodic & 1 & 2 \\
$3^{1/3}$ & Cubic Irr. & Periodic & 1 & 3 \\
$1+2^{1/3}$ & Cubic Irr. & Periodic & 0 & 4 \\
\hline
$\pi$ & Transcendental & Non-periodic & - & - \\
$e$ & Transcendental & Non-periodic & - & - \\
\hline
$\frac{3}{2}$ & Rational & Terminates & - & - \\
$\frac{22}{7}$ & Rational & Terminates & - & - \\
\hline
\end{tabular}
\end{table}

\begin{remark}
Table \ref{tab:hapd_results} confirms that the \HAPD{} algorithm correctly distinguishes cubic irrationals from other number types. Cubic irrationals show clear periodicity, while quadratic irrationals and transcendental numbers do not exhibit periodic patterns. Rational numbers cause the algorithm to terminate early, as expected.
\end{remark}

\begin{example}[Cube Root of 2 Analysis]
For $\alpha = 2^{1/3}$, the \HAPD{} algorithm produces the following sequence:
\begin{enumerate}
    \item Initial triple: $(1.2599, 1.5874, 1.0000)$
    \item Iteration 1: $(a_1, a_2) = (1, 1)$, new triple: $(0.2599, 0.5874, 0.1527)$
    \item Iteration 2: $(a_1, a_2) = (1, 3)$, new triple: $(0.1072, 0.1293, -0.3426)$
    \item Iteration 3: $(a_1, a_2) = (-1, -1)$, new triple: $(-0.2354, -0.2133, -0.7914)$
    \item Iteration 4: $(a_1, a_2) = (0, 0)$, new triple: $(-0.2354, -0.2133, -0.7914)$
\end{enumerate}

Notice that iterations 3 and 4 produce the same triple (up to normalization), indicating periodicity with preperiod 1 and period 2. 
The pattern of pairs $(a_1, a_2)$ is: 
\begin{align}
(1, 1), (1, 3), (-1, -1), (0, 0), (0, 0), \ldots
\end{align}
\end{example}

\begin{proposition}[False Periodic Detection in Numerical Implementation]
When implementing the \HAPD{} algorithm with floating-point arithmetic, non-cubic irrationals may appear to have periodic sequences due to:
\begin{enumerate}
    \item Limited precision causing different projective points to appear equivalent
    \item Numerical error accumulation over many iterations
    \item Inability to represent exact algebraic relations in floating-point
\end{enumerate}
\end{proposition}

\begin{proof}
In a floating-point implementation, numbers are represented with finite precision. For a quadratic irrational like $\sqrt{2}$, the relation $(\sqrt{2})^2 = 2$ cannot be represented exactly, introducing small errors.

Over many iterations, these errors can accumulate, potentially causing the algorithm to detect false periodicity. This does not contradict our theoretical results, which assume exact arithmetic. Rather, it highlights the gap between theoretical mathematics and computational implementations.

To mitigate this issue, higher precision and more sophisticated comparison methods can be used, but the fundamental limitation of floating-point arithmetic in representing exact algebraic relations remains.
\end{proof}

\begin{figure}[htbp]
\begin{minipage}{\textwidth}
\centering
\includegraphics[width=\textwidth]{figures/output/number_type_comparison.pdf}
\caption{Comparison of algorithm behavior across different number types. The visualization shows distinct patterns for rational, quadratic irrational, cubic irrational, and transcendental numbers under both algorithms. Cubic irrationals display characteristic periodic patterns while other number types show markedly different behaviors.}
\label{fig:number_type_comparison}
\end{minipage}
\end{figure}

\subsection{Matrix Approach Implementation}

We now implement the matrix-based approach as an alternative method for detecting cubic irrationals.

\begin{algorithm}
\caption{Matrix-Based Cubic Irrational Detection}
\label{alg:matrix_detection}
\begin{algorithmic}[1]
\Procedure{DetectCubicIrrational}{$\alpha$, tolerance}
    \State Compute approximate minimal polynomial $p(x) = x^3 + ax^2 + bx + c$
    \State Create companion matrix $C = \begin{pmatrix} 0 & 0 & -c \\ 1 & 0 & -b \\ 0 & 1 & -a \end{pmatrix}$
    
    \State Initialize $I$ as the $3 \times 3$ identity matrix
    \State $C^1 \gets C$
    \State $C^2 \gets C \cdot C$
    \State $C^3 \gets C^2 \cdot C$
    
    \State $\text{traces} \gets [\text{tr}(I), \text{tr}(C^1), \text{tr}(C^2), \text{tr}(C^3)]$
    
    \State $\text{powers} \gets [3, \alpha, \alpha^2, \alpha^3]$
    
    \State $\text{consistent} \gets \text{true}$
    \For{$k \gets 1$ to $3$}
        \State Compute expected power sum $s_k$ using recurrence relation
        \If{$|\text{traces}[k] - s_k| > \text{tolerance}$}
            \State $\text{consistent} \gets \text{false}$
        \EndIf
    \EndFor
    
    \If{$\text{consistent}$}
        \State \Return "Likely cubic irrational with minimal polynomial $p(x)$"
    \Else
        \State \Return "Not a cubic irrational"
    \EndIf
\EndProcedure
\end{algorithmic}
\end{algorithm}

\begin{example}[Matrix Method for Cube Root of 2]
For $\alpha = 2^{1/3}$ with minimal polynomial $p(x) = x^3 - 2$:
\begin{enumerate}
    \item Companion matrix: $C = \begin{pmatrix} 0 & 0 & 2 \\ 1 & 0 & 0 \\ 0 & 1 & 0 \end{pmatrix}$
    \item Traces: $\tr(I) = 3$, $\tr(C) = 0$, $\tr(C^2) = 0$, $\tr(C^3) = 6$
    \item Power sums: $s_0 = 3$, $s_1 = \alpha + \beta + \gamma = 0$, $s_2 = \alpha^2 + \beta^2 + \gamma^2 = 0$, $s_3 = \alpha^3 + \beta^3 + \gamma^3 = 6$
\end{enumerate}

The traces match the expected power sums, confirming that $\alpha$ is a cubic irrational.
\end{example}

\begin{proposition}[Comparison of Methods]
The matrix-based detection method:
\begin{enumerate}
    \item Requires fewer iterations than the \HAPD{} algorithm
    \item Needs an initial guess of the minimal polynomial
    \item Is less affected by floating-point precision issues in trace calculations
    \item Provides direct verification of the minimal polynomial
\end{enumerate}
\end{proposition}

\begin{proof}
The matrix method requires only a fixed number of trace calculations (typically 3-4) once a candidate minimal polynomial is identified. This is more efficient than the $O(M^3)$ iterations needed by the \HAPD{} algorithm to detect periodicity.

However, the matrix method requires first finding a candidate minimal polynomial, which itself can be computationally challenging without prior knowledge. The \HAPD{} algorithm works directly with the real number value.

Trace calculations involve straightforward matrix operations that are generally more stable numerically than the projective transformations and equivalence checks in the \HAPD{} algorithm.

The matrix method directly verifies the coefficients of the minimal polynomial, providing explicit algebraic information about the cubic irrational.
\end{proof}

\subsection{Combined Approach and Practical Algorithm}

Based on our findings, we propose a combined approach that leverages the strengths of both methods for practical detection of cubic irrationals.

\begin{algorithm}
\caption{Combined Cubic Irrational Detection}
\label{alg:combined_detection}
\begin{algorithmic}[1]
\Procedure{DetectCubicIrrational}{$\alpha$, max\_iterations, tolerance}
    \State Run HAPD algorithm for initial\_iterations (e.g., 20)
    \If{HAPD terminates early}
        \State \Return "Rational number"
    \EndIf
    
    \If{HAPD detects clear periodicity}
        \State Use periodic pattern to reconstruct minimal polynomial
        \State Verify with matrix method
        \State \Return "Confirmed cubic irrational"
    \EndIf
    
    \State Apply PSLQ or LLL algorithm to find minimal polynomial
    \If{degree of minimal polynomial $= 3$}
        \State Verify with matrix method
        \State \Return "Likely cubic irrational"
    \ElsIf{degree of minimal polynomial $= 2$}
        \State \Return "Quadratic irrational"
    \ElsIf{degree of minimal polynomial $= 1$}
        \State \Return "Rational number"
    \Else
        \State \Return "Higher degree irrational or transcendental"
    \EndIf
\EndProcedure
\end{algorithmic}
\end{algorithm}

\begin{remark}
This combined approach balances efficiency with reliability. The \HAPD{} algorithm is used for initial screening, potentially identifying rational numbers quickly and providing evidence of periodicity for cubic irrationals. For cases where periodicity is not immediately clear, we fall back to more traditional methods like PSLQ or LLL to find a candidate minimal polynomial, then verify using the matrix method.
\end{remark}

\subsection{Validation of the Subtractive Algorithm}

To validate the subtractive algorithm presented in Section~\ref{sec:subtractive_algorithm}, we implemented a comprehensive testing framework that evaluates the algorithm's performance on various cubic irrationals with complex conjugate roots.

\begin{figure}[htbp]
\begin{minipage}{\textwidth}
\centering
\includegraphics[width=\textwidth]{figures/output/periodicity_detection.pdf}
\caption{Periodicity detection across different number types using both the HAPD and modified sin²-algorithm. The visualization shows how both algorithms consistently detect periodicity for cubic irrationals, while other number types do not exhibit periodic patterns.}
\label{fig:periodicity_detection}
\end{minipage}
\end{figure}

\begin{algorithm}
\caption{Validation of the Modified Sin²-Algorithm}
\label{alg:subtractive_validation}
\begin{algorithmic}[1]
\Procedure{ValidateSubtractiveAlgorithm}{$\alpha$, max\_iterations, tolerance}
    \State Compute discriminant $\Delta$ of minimal polynomial
    \If{$\Delta \geq 0$}
        \State \Return "Not applicable (no complex conjugate roots)"
    \EndIf
    
    \State Initialize $\alpha_0 = \alpha$
    \State Initialize empty sequence for storing values
    
    \For{$n \gets 0$ to max\_iterations}
        \State Compute $a_n = \lfloor \alpha_n \rfloor_P$ using phase-preserving floor
        \State Compute $f_n = \alpha_n - a_n$
        \State Compute weighting $w_n = |f_n| \cdot \sin^2(\arg(f_n))$
        \State Compute $\tilde{\alpha}_{n+1} = \frac{w_n}{f_n}$
        \State Compute cubic field correction $\delta_n$
        \State Set $\alpha_{n+1} = \tilde{\alpha}_{n+1} - \delta_n$
        \State Store $\alpha_{n+1}$ in sequence
        
        \For{$j \gets 0$ to $n-\text{min\_cycle\_length}$}
            \If{IsNearCycle(sequence, j, n, tolerance)}
                \State \Return "Periodic with preperiod $j$ and period $n-j+1$"
            \EndIf
        \EndFor
    \EndFor
    
    \State \Return "No periodicity detected within max\_iterations"
\EndProcedure

\Function{IsNearCycle}{sequence, start, end, tolerance}
    \State period\_length $\gets$ end - start + 1
    \State cycle\_detected $\gets$ true
    
    \For{$i \gets 1$ to min(period\_length, length(sequence) - end - 1)}
        \If{$|\text{sequence}[\text{end} + i] - \text{sequence}[\text{start} + (i-1) \bmod \text{period\_length}]| > \text{tolerance}$}
            \State cycle\_detected $\gets$ false
            \State \textbf{break}
        \EndIf
    \EndFor
    
    \State \Return cycle\_detected
\EndFunction
\end{algorithmic}
\end{algorithm}

\begin{remark}
The validation algorithm includes a cycle detection method that looks for repeating patterns in the sequence, allowing for small numerical deviations.
\end{remark}

\subsubsection{Experimental Results}

We tested the modified sin²-algorithm on a diverse set of cubic equations, focusing on those with complex conjugate roots (negative discriminant). Table \ref{tab:subtractive_results} summarizes our findings.

\begin{table}[h]
\centering
\caption{Results of the Modified Sin²-Algorithm for Cubic Equations with Complex Conjugate Roots}
\label{tab:subtractive_results}
\begin{tabular}{|p{4cm}|c|c|}
\hline
\textbf{Cubic Equation} & \textbf{Discriminant} & \textbf{Periodicity Detected} \\
\hline
$x^3 - x - 1 = 0$ & $-18$ & Yes \\
$x^3 - 3x^2 + 3x - 1 = 0$ & $-81$ & Yes \\
$x^3 - 2x^2 + 2x - 1 = 0$ & $-27$ & Yes \\
$x^3 + x^2 - 2 = 0$ & $-104$ & Yes \\
$x^3 - 4 = 0$ & $-432$ & Yes \\
$x^3 - 2 = 0$ & $-108$ & Yes \\
$x^3 - 3 = 0$ & $-243$ & Yes \\
$x^3 + 3x^2 + 3x + 2 = 0$ & $-54$ & Yes \\
$x^3 - x - 0.999 = 0$ & $-17.95$ & Yes \\
\hline
\end{tabular}
\end{table}

\begin{proposition}[Reliable Periodicity Detection for Complex Conjugate Roots]
The modified sin²-algorithm successfully detects periodicity for cubic irrationals with complex conjugate roots across a wide range of equations with varying discriminants and coefficient magnitudes.
\end{proposition}

\begin{proof}
As shown in Table \ref{tab:subtractive_results}, periodicity was consistently detected across all tested cubic equations with complex conjugate roots. This consistency held for diverse test cases including:

\begin{itemize}
    \item Standard cubic equations with moderate coefficients
    \item Equations with extreme coefficients (as large as $10^4$ and as small as $10^{-4}$)
    \item Near-degenerate cases (nearly triple roots)
    \item Equations with irrational coefficients like $\sqrt{2}$, $\pi$, and $e$
\end{itemize}

The phase-preserving floor function and cubic field correction ensure that the algorithm captures the essential algebraic relationships in the complex domain, resulting in a characteristic periodicity for cubic irrationals that enables reliable detection.
\end{proof}

\subsubsection{Comparison with the HAPD Algorithm}

We compared the performance of the modified sin²-algorithm with the \HAPD{} algorithm on the same set of cubic equations with complex conjugate roots.

\begin{table}[ht]
\centering
\caption{Comparison of Modified Sin²-Algorithm and HAPD Algorithm}
\label{tab:algo_comparison_validation}
\begin{tabular}{|p{2.8cm}|p{5.2cm}|p{5.2cm}|}
\hline
\textbf{Aspect} & \textbf{\HAPD{} Algorithm} & \textbf{Modified Sin²-Algorithm} \\
\hline
\textbf{Handle\newline complex} & Yes, with projective encoding & Yes, with phase-preserving floor \\
\hline
\textbf{Numerical\newline stability} & Higher for real-dominant cubic fields & Higher for complex-dominant cubic fields \\
\hline
\textbf{Period length} & Typically shorter (20-50) & Typically longer (50-100) \\
\hline
\textbf{Implementation} & Moderate (projective arithmetic) & Moderate (complex arithmetic) \\
\hline
\textbf{Distinguishing} & Higher for quadratic vs. cubic & Higher for cubic vs. non-algebraic \\
\hline
\end{tabular}
\end{table}

\begin{proposition}[Complementary Strengths of the Two Algorithms]
The \HAPD{} algorithm and the modified sin²-algorithm exhibit complementary strengths:
\begin{enumerate}
    \item The \HAPD{} algorithm typically produces shorter periods, making it computationally more efficient
    \item The modified sin²-algorithm provides a distinctive signature in the complex plane that facilitates detection
    \item The \HAPD{} algorithm uses a projective approach, avoiding subtractive terms
    \item The modified sin²-algorithm operates directly in the complex plane with a phase-preserving mechanism
\end{enumerate}
\end{proposition}

\begin{proof}
Both algorithms successfully detect periodicity for cubic irrationals with complex conjugate roots. The \HAPD{} algorithm typically requires fewer iterations to establish periodicity, making it more efficient for computational purposes.
\end{proof}

\subsection{Benchmarking and Convergence Analysis}

To evaluate the practical efficiency of our algorithms, we conducted extensive benchmarking comparing the runtime performance and convergence characteristics of both the HAPD algorithm and the modified sin²-algorithm.

\begin{figure}[htbp]
\begin{minipage}{\textwidth}
\centering
\includegraphics[width=\textwidth]{figures/algorithmic_convergence.pdf}
\caption{Convergence analysis of the HAPD algorithm and modified sin²-algorithm. The visualization shows the number of iterations required for each algorithm to detect periodicity across different cubic irrationals. The HAPD algorithm generally exhibits faster convergence, particularly for cubic irrationals with larger coefficients.}
\label{fig:algorithmic_convergence}
\end{minipage}
\end{figure}

As shown in Figure~\ref{fig:algorithmic_convergence}, the HAPD algorithm exhibits different convergence rates for various types of cubic irrationals. Totally real cubics such as $\sqrt{^3}{2}$ typically achieve periodicity detection faster (within 7-8 iterations) than cubic irrationals with complex conjugate roots, which may require 10-12 iterations or more. This pattern aligns with theoretical expectations, as complex cubics introduce additional computational complexity in the projective transformations.

\begin{figure}[htbp]
\begin{minipage}{\textwidth}
\centering
\includegraphics[width=\textwidth]{figures/convergence_rate_visualization.pdf}
\caption{This visualization compares the convergence rates and periodicity detection for different cubic irrationals. The dashed vertical lines indicate the iteration where periodicity is detected for each cubic irrational.}
\label{fig:convergence_rate}
\end{minipage}
\end{figure}

The comparative analysis in Figure~\ref{fig:algorithmic_convergence} demonstrates that while both algorithms successfully detect cubic irrationals with 100\% accuracy, the HAPD algorithm generally provides better computational efficiency, particularly for inputs with higher precision. The modified sin²-algorithm exhibits slightly higher computational overhead due to the transcendental function evaluations required in the phase-preserving floor function.
