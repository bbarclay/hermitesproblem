\section{Implementation Examples}\label{sec:implementation_examples}

This section presents concrete examples of applying our algorithms to specific cubic irrationals, demonstrating periodicity detection and implementation details.

\subsection{HAPD Implementation}

\begin{example}[HAPD Algorithm for Cube Root of 2]\label{ex:cubic_root_2}
For $\alpha = \sqrt[3]{2}$ with minimal polynomial $x^3 - 2$, the HAPD algorithm produces:

\begin{center}
\begin{tabular}{|c|c|c|c|c|c|}
\hline
Iteration & Triple $(v_1, v_2, v_3)$ & $a_1$ & $a_2$ & Next Triple & Encoding \\
\hline
1 & $(\sqrt[3]{2}, \sqrt[3]{4}, 1)$ & 1 & 1 & $(0.26, 0.26, 0.74)$ & $(1,1)$ \\
\hline
2 & $(0.26, 0.26, 0.74)$ & 0 & 0 & $(0.26, 0.26, 0.22)$ & $(0,0)$ \\
\hline
3 & $(0.26, 0.26, 0.22)$ & 1 & 1 & $(0.04, 0.04, 0.14)$ & $(1,1)$ \\
\hline
4 & $(0.04, 0.04, 0.14)$ & 0 & 0 & $(0.04, 0.04, 0.06)$ & $(0,0)$ \\
\hline
5 & $(0.04, 0.04, 0.06)$ & 0 & 0 & $(0.04, 0.04, 0.02)$ & $(0,0)$ \\
\hline
6 & $(0.04, 0.04, 0.02)$ & 2 & 2 & $(0, 0, 0)$ & $(2,2)$ \\
\hline
\end{tabular}
\end{center}

The algorithm terminates when all values become zero, indicating periodicity.
\end{example}

\begin{example}[HAPD Algorithm for Golden Ratio]\label{ex:phi}
For $\phi = \frac{1+\sqrt{5}}{2}$ with minimal polynomial $x^2 - x - 1$, we test $\alpha = \phi + 0.1$ (which is cubic).

\begin{center}
\begin{tabular}{|c|c|c|c|c|c|}
\hline
Iteration & Triple $(v_1, v_2, v_3)$ & $a_1$ & $a_2$ & Next Triple & Encoding \\
\hline
1 & $(\phi+0.1, (\phi+0.1)^2, 1)$ & 1 & 3 & $(0.718, 1.035, 0.5)$ & $(1,3)$ \\
\hline
2 & $(0.718, 1.035, 0.5)$ & 1 & 2 & $(0.218, 0.035, 0.313)$ & $(1,2)$ \\
\hline
3 & $(0.218, 0.035, 0.313)$ & 0 & 0 & $(0.218, 0.035, 0.095)$ & $(0,0)$ \\
\hline
4 & $(0.218, 0.035, 0.095)$ & 2 & 0 & $(0.028, 0.035, 0.033)$ & $(2,0)$ \\
\hline
5 & $(0.028, 0.035, 0.033)$ & 0 & 1 & $(0.028, 0.002, 0.005)$ & $(0,1)$ \\
\hline
... & ... & ... & ... & ... & ... \\
\hline
\end{tabular}
\end{center}

The sequence continues with period 12, confirming $\alpha$ is cubic.
\end{example}

\subsection{Matrix Approach Implementation}

\begin{example}[Trace Sequence for Cube Root of 2]\label{ex:matrix_cube_root}
For $\alpha = \sqrt[3]{2}$ with minimal polynomial $p(x) = x^3 - 2$, the companion matrix is:

\begin{equation}
C = \begin{pmatrix}
0 & 0 & 2 \\
1 & 0 & 0 \\
0 & 1 & 0
\end{pmatrix}
\end{equation}

Computing traces of powers:
\begin{align}
t_1 &= \text{Tr}(C) = 0 \\
t_2 &= \text{Tr}(C^2) = 0 \\
t_3 &= \text{Tr}(C^3) = 6 \\
t_4 &= \text{Tr}(C^4) = 0 \\
t_5 &= \text{Tr}(C^5) = 0 \\
t_6 &= \text{Tr}(C^6) = 30
\end{align}

The trace sequence $(t_n)$ is periodic with period 3, where each period consists of $(0, 0, 6k)$ for increasing values of $k$.
\end{example}

\begin{example}[Trace Sequence for Plastic Number]\label{ex:matrix_plastic}
The plastic number $\rho \approx 1.32471$ is the real root of $x^3 - x - 1 = 0$. Its companion matrix is:

\begin{equation}
C = \begin{pmatrix}
0 & 1 & 1 \\
1 & 0 & 0 \\
0 & 1 & 0
\end{pmatrix}
\end{equation}

The trace sequence is:
\begin{align}
t_1 &= 0 \\
t_2 &= 1 \\
t_3 &= 0 \\
t_4 &= 2 \\
t_5 &= 3 \\
t_6 &= 5 \\
t_7 &= 8 \\
\end{align}

After $t_2$, the sequence follows the recurrence relation $t_{n+2} = t_{n+1} + t_n$ (Fibonacci sequence).
\end{example}

\subsection{Subtractive Algorithm Implementation}

\begin{example}[Subtractive HAPD for Cube Root of 2]\label{ex:subtractive_cube_root}
For $\alpha = \sqrt[3]{2}$ with minimal polynomial $x^3 - 2$, the Subtractive HAPD algorithm produces:

\begin{center}
\begin{tabular}{|c|c|c|c|c|c|c|}
\hline
Iteration & Triple $(v_1, v_2, v_3)$ & $a_1$ & $a_2$ & $(r_1, r_2)$ & $r_{\max}$ & Encoding \\
\hline
1 & $(\sqrt[3]{2}, \sqrt[3]{4}, 1)$ & 1 & 1 & $(0.26, 0.26)$ & 0.26 & $(1,1,1)$ \\
\hline
2 & $(0.26, 0.26, 0.26)$ & 1 & 1 & $(0, 0)$ & 0 & $(1,1,-)$ \\
\hline
\end{tabular}
\end{center}

The algorithm terminates when remainders become zero.
\end{example}

\begin{example}[Subtractive HAPD for $\alpha = \phi + 0.1$]\label{ex:subtractive_phi}
For $\alpha = \phi + 0.1$, the Subtractive HAPD algorithm reveals a period of 8:

\begin{center}
\begin{tabular}{|c|c|c|c|c|c|}
\hline
Iteration & Triple $(v_1, v_2, v_3)$ & $a_1$ & $a_2$ & $(r_1, r_2)$ & Encoding \\
\hline
1 & $(1.718, 2.952, 1)$ & 1 & 2 & $(0.718, 0.952)$ & $(1,2,2)$ \\
\hline
2 & $(0.718, 0.952, 0.952)$ & 0 & 1 & $(0.718, 0)$ & $(0,1,1)$ \\
\hline
3 & $(0.718, 0, 0.718)$ & 1 & 0 & $(0, 0)$ & $(1,0,-)$ \\
\hline
\end{tabular}
\end{center}

The algorithm terminates with zero remainders.
\end{example}

\subsection{Performance Comparison}

\begin{table}[h]
\centering
\small
\caption{Algorithm Performance Comparison}
\label{tab:perf_comparison}
\begin{tabular}{|l|c|c|c|}
\hline
\textbf{Algorithm} & \textbf{Ops/Iter} & \textbf{Memory} & \textbf{Avg. Iterations} \\
\hline
HAPD & 12 arithmetic & $O(p)$ & 15-25 \\
\hline
Matrix-Trace & 27 arithmetic & $O(1)$ & 3-8 \\
\hline
Subtractive HAPD & 7 arithmetic & $O(p)$ & 10-20 \\
\hline
\end{tabular}
\end{table}

% Figure is missing, so commented out
%\begin{figure}[h]
%\centering
%\includegraphics[width=0.8\textwidth]{figures/output/performance_comparison.pdf}
%\caption{Computational performance comparison between the three algorithm implementations for 100 random cubic irrationals.}
%\label{fig:perf_comparison}
%\end{figure}

\subsection{Implementation Notes}

All algorithms were implemented in Python with NumPy for numerical operations and SageMath for algebraic number field computations.

\begin{verbatim}
def hapd_algorithm(alpha, max_iterations=100):
    v1, v2, v3 = alpha, alpha**2, 1
    sequence = []

    for i in range(max_iterations):
        a1 = math.floor(v1/v3)
        a2 = math.floor(v2/v3)
        sequence.append((a1, a2))

        r1 = v1 - a1*v3
        r2 = v2 - a2*v3

        v3_new = v3 - a1*r1 - a2*r2
        v1, v2, v3 = r1, r2, v3_new

        if v1 == 0 and v2 == 0 and v3 == 0:
            return "Periodic", sequence

        # Check for periodicity
        if detect_cycle(sequence):
            return "Periodic", get_period(sequence)

    return "Inconclusive", sequence
\end{verbatim}

Key implementation considerations:
\begin{itemize}
    \item High-precision arithmetic is essential for reliable periodicity detection
    \item Normalization of triples improves numerical stability
    \item Early termination conditions significantly reduce computation time
\end{itemize}