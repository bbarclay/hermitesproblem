\begin{abstract}
Hermite's problem seeks an algorithm that characterizes cubic irrationals through periodicity, analogous to how continued fractions identify quadratic irrationals. We present a complete solution through three complementary approaches: (1) the Hermite Algorithm for Periodicity Detection (HAPD) operating in projective space, (2) a matrix-based characterization using companion matrices and trace sequence periodicity, and (3) a modified $\sin^2$-algorithm that handles complex conjugate roots via a phase-preserving floor function. Each method produces eventually periodic sequences precisely for cubic irrationals, including those with complex conjugate roots—previously an unsolved case. We rigorously prove the correctness of each approach, establish their mathematical equivalence, and provide comprehensive numerical validation. Our work creates a unified framework connecting periodicity to algebraic degree for cubic irrationals, resolving a long-standing problem in Diophantine approximation.

\textbf{Keywords:} Cubic irrationals, Hermite's problem, continued fractions, projective geometry, companion matrices, trace sequences, Diophantine approximation
\end{abstract}

An interactive version of this paper can be found at https://github.com/bbarclay/hermitesproblem. 

