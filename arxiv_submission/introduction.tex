\section{Introduction}\label{sec:intro}

Hermite's original problem, raised in correspondence with Jacobi in 1848 \cite{Hermite1848}, asked for a natural generalization of continued fractions that would characterize cubic irrationals through periodicity. For quadratic irrationals, the answer is known—continued fractions produce an eventually periodic sequence if and only if a number is quadratic irrational. The cubic case has remained unresolved for cubic irrationals with complex conjugate roots.

Four main approaches have been previously attempted:
\begin{itemize}
\item Jacobi-Perron algorithm (1868): higher-dimensional but fails for complex conjugate roots
\item Brun's algorithm (1920): modified Jacobi-Perron with similar limitations
\item Poincaré's geometric approach: lacks consistent periodicity properties
\item Karpenkov's sin²-algorithm: demonstrated periodicity for totally-real cubic irrationals only
\end{itemize}

We resolve Hermite's problem by developing:
\begin{enumerate}
\item The Hermite Algorithm for Periodicity Detection (HAPD) operating in projective space, producing eventually periodic sequences if and only if the input is a cubic irrational
\item A modified sin²-algorithm with phase-preserving floor function handling complex conjugate roots
\end{enumerate}

This paper is organized as follows:
\begin{itemize}
\item Section \ref{sec:galois_theory}: proof that cubic irrationals cannot have periodic continued fraction expansions
\item Section \ref{sec:hapd_algorithm}: the HAPD algorithm and its theoretical foundation
\item Section \ref{sec:matrix_approach}: matrix-based characterization via companion matrices
\item Section \ref{sec:matrix_verification}: enhanced matrix verification method
\item Section \ref{sec:equivalence}: equivalence between algorithmic and matrix approaches
\item Section \ref{sec:subtractive_algorithm}: modified sin²-algorithm
\item Section \ref{sec:numerical_validation}: numerical validation across different number classes
\item Section \ref{sec:objections}: theoretical objections and edge cases
\item Section \ref{sec:conclusion}: implications and generalizations
\end{itemize}
