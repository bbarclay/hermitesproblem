\section{Introduction}\label{sec:intro}

Hermite's problem, posed to Jacobi in 1848 \cite{Hermite1848}, sought a generalization of continued fractions that would characterize cubic irrationals through periodicity. Continued fractions produce eventually periodic sequences precisely for quadratic irrationals, but the cubic case with complex conjugate roots remained unsolved.

Previous approaches include:
\begin{itemize}
\item Jacobi-Perron algorithm (1868) \cite{Jacobi1868}: fails for complex conjugate roots
\item Brun's algorithm (1920) \cite{Brentjes1981}: similar limitations
\item Poincaré's geometric approach \cite{KarpenkovBook}: lacks consistent periodicity
\item Karpenkov's $\sin^2$-algorithm \cite{Karpenkov2019}: works only for totally-real cubics
\end{itemize}

We resolve Hermite's problem through three novel approaches:
\begin{enumerate}
\item HAPD algorithm in projective space, producing periodic sequences if and only if the input is cubic irrational
\item Matrix characterization using companion matrices and trace sequences with modular periodicity
\item Modified $\sin^2$-algorithm handling complex conjugate roots via phase-preserving floor functions
\end{enumerate}

Contents:
\begin{itemize}
\item \S\ref{sec:galois_theory}: proof of continued fraction non-periodicity
\item \S\ref{sec:hapd_algorithm}: HAPD algorithm foundations
\item \S\ref{sec:matrix_approach}: matrix characterization via companion matrices
% Removed reference to matrix-verification as it's merged
\item \S\ref{sec:equivalence}: equivalence between approaches
\item \S\ref{sec:subtractive_algorithm}: modified $\sin^2$-algorithm
\item \S\ref{sec:numerical_validation}: numerical validation
\item \S\ref{sec:implementation_examples}: implementation examples
\item \S\ref{sec:objections}: addressing theoretical objections
\item \S\ref{sec:conclusion}: implications and generalizations
\end{itemize}

\textbf{Computational Approach}. Our work combines theoretical insights with practical verification, offering a computational framework for exploring cubic irrationals (Section~\ref{sec:matrix_approach}). We develop algorithms that determine whether a given real number is cubic irrational based on the periodicity of its HAPD sequence. These algorithms are implemented and tested with various inputs, providing empirical validation of the theoretical results.
