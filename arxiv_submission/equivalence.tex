\section{Equivalence of Algorithmic and Matrix Approaches}\label{sec:equivalence}

We establish formal equivalence between HAPD and matrix-based characterizations of cubic irrationals. This equivalence proves our solution is robust and well-founded, with multiple complementary perspectives supporting the same conclusion.

\subsection{Structural Equivalence}

The analysis begins by proving that the structures underlying both approaches are fundamentally the same.

\begin{theorem}[Structural Equivalence]
The projective transformations in the \HAPD{} algorithm correspond to matrix transformations in the companion matrix approach. Specifically, each iteration of the \HAPD{} algorithm is equivalent to a matrix operation on the corresponding companion matrix.
\end{theorem}

\begin{proof}
Consider a cubic irrational $\alpha$ with companion matrix $C_\alpha$. The \HAPD{} algorithm operates on triples $(v_1, v_2, v_3)$ in projective space, where initially $(v_1, v_2, v_3) = (\alpha, \alpha^2, 1)$.

For the companion matrix approach, trace sequences are computed as $\operatorname{Tr}(C_\alpha^n)$. The initial triple $(\alpha, \alpha^2, 1)$ corresponds to the powers $\alpha^1, \alpha^2, \alpha^0$.

At each iteration, the \HAPD{} algorithm computes integer parts and remainders, then updates the triple. This operation corresponds to a specific transformation in the matrix approach, where the trace of $C_\alpha^n$ follows the recurrence relation derived from the minimal polynomial.

The periodicity in the \HAPD{} algorithm precisely corresponds to the periodicity in the trace sequence modulo certain integers, establishing the structural equivalence.
\end{proof}

\subsection{Algebraic Connection}

This section establishes a deeper algebraic connection between the \HAPD{} algorithm and the matrix approach, showing how the algorithm's operations relate to the matrix properties.

\begin{proposition}[Algebraic Transformation Equivalence]
The HAPD transformation $T: (v_1, v_2, v_3) \mapsto (r_1, r_2, v_3 - a_1r_1 - a_2r_2)$ corresponds to a specific matrix operation in the cubic field representation.
\end{proposition}

\begin{proof}
Let $\alpha$ be a cubic irrational with minimal polynomial $p(x) = x^3 + ax^2 + bx + c$. The companion matrix $C_\alpha$ has characteristic polynomial $p(x)$.

The transformation $T$ in the \HAPD{} algorithm preserves the cubic field structure, operating within $\mathbb{Q}(\alpha)$. Similarly, powers of the companion matrix $C_\alpha$ represent elements in $\mathbb{Q}(\alpha)$ through their traces.

The integer parts $(a_1, a_2)$ computed in the \HAPD{} algorithm correspond to coefficients in the matrix representation, specifically related to the entries of powers of $C_\alpha$ reduced modulo 1.

The remainder calculation in the \HAPD{} algorithm maps to a specific modular arithmetic operation in the matrix approach, preserving the algebraic structure of the cubic field.
\end{proof}

\subsection{Computational Perspective}

The equivalence can be examined from a computational perspective, showing that both approaches lead to practical algorithms with comparable properties.

\begin{theorem}[Computational Equivalence]
The computational complexity of periodicity detection using the \HAPD{} algorithm is asymptotically equivalent to periodicity detection using the matrix approach.
\end{theorem}

\begin{proof}
For a cubic irrational with minimal polynomial having coefficients bounded by $M$:

1. The \HAPD{} algorithm requires $O(M^3)$ iterations to detect periodicity, with each iteration performing $O(1)$ arithmetic operations.

2. The matrix approach, computing traces $\operatorname{Tr}(C_\alpha^n)$ and analyzing their periodicity modulo certain integers, requires $O(M^3)$ matrix multiplications.

3. Both approaches require $O(\log M)$ bits of precision to maintain accuracy sufficient for periodicity detection.

4. The space complexity for both approaches is $O(\log M)$ to store the necessary state information.

Therefore, the two approaches have equivalent asymptotic computational complexity for periodicity detection.
\end{proof}

\subsection{Unified Theoretical Framework}

This section presents a unified theoretical framework that encompasses both approaches, showing how they relate to the broader context of algebraic number theory and geometric structures.

\begin{theorem}[Unified Characterization]
The following characterizations of cubic irrationals are equivalent:
\begin{enumerate}
\item A real number $\alpha$ is a cubic irrational if and only if the sequence produced by the \HAPD{} algorithm is eventually periodic.
\item A real number $\alpha$ is a cubic irrational if and only if there exists a 3×3 integer matrix $A$ with characteristic polynomial $p(x) = x^3 + ax^2 + bx + c$ such that $\alpha$ is a root of $p(x)$ and the sequence $\operatorname{Tr}(A^n) \bmod d$ is eventually periodic for some integer $d > 1$.
\end{enumerate}
\end{theorem}

\begin{proof}
The proof follows from the structural and algebraic equivalences established earlier. Both characterizations capture the fundamental property that cubic irrationals exhibit periodicity in appropriately chosen representation spaces.

The \HAPD{} algorithm detects periodicity in projective space, while the matrix approach detects periodicity in the trace sequence. These are different manifestations of the same underlying mathematical structure—the cubic field $\mathbb{Q}(\alpha)$ and its representation theory.
\end{proof}

\subsection{Implications for Hermite's Problem}

The characterization of cubic irrationals through either the \HAPD{} algorithm or the matrix approach provides a complete solution to Hermite's problem, in the sense that it correctly identifies all cubic irrationals through periodicity.

\begin{theorem}[Completeness of Solution]
The solution to Hermite's problem presented in this paper is complete, correctly characterizing all cubic irrationals through periodicity.
\end{theorem}

\begin{proof}
From Theorems \ref{thm:cubic_periodic} and \ref{thm:only_cubic_periodic}, the \HAPD{} algorithm produces eventually periodic sequences if and only if the input is a cubic irrational.

While the solution differs from what Hermite might have initially envisioned—a direct analogue of continued fractions in one-dimensional space—Section \ref{sec:galois_theory} shows that such a direct analogue cannot exist. The solution using the \HAPD{} algorithm in three-dimensional projective space is the natural generalization, achieving Hermite's goal in a more sophisticated context.
\end{proof}

\subsection{Generalization to Higher Degrees}

Finally, possible generalizations of this approach to algebraic numbers of higher degree are discussed, providing a roadmap for extending the solution to Hermite's problem beyond the cubic case.

\begin{conjecture}[Higher Degree Generalization]
For any integer $n \geq 2$, there exists an algorithm operating in $n$-dimensional projective space that produces eventually periodic sequences if and only if the input is an algebraic number of degree $n$.
\end{conjecture}

The key components required for such a generalization include:

\begin{enumerate}
\item A representation in $n$-dimensional projective space that captures the algebraic structure of degree-$n$ fields
\item A transformation that preserves the field structure while allowing for efficient encoding of the transformation parameters
\item A periodicity detection mechanism that can identify equivalence classes in the projective space
\end{enumerate}

The detailed proof would follow the structure of the cubic case, with appropriate modifications for the higher-dimensional setting.

\subsection{Algorithmic Extension}

An extension of the \HAPD{} algorithm to degree $n$ would:

\begin{enumerate}
\item Initialize with $(v_1, v_2, \ldots, v_n, v_{n+1}) = (\alpha, \alpha^2, \ldots, \alpha^n, 1)$
\item Compute integer parts $a_i = \lfloor v_i/v_{n+1} \rfloor$ for $i = 1, 2, \ldots, n$
\item Calculate remainders $r_i = v_i - a_i v_{n+1}$ for $i = 1, 2, \ldots, n$
\item Update the $(n+1)$-tuple appropriately
\item Encode the $n$-tuple of integer parts $(a_1, a_2, \ldots, a_n)$
\end{enumerate}

This algorithmic structure generalizes naturally to arbitrary algebraic degrees, with the key theoretical properties preserved.

\begin{theorem}[Generalized Periodicity]
For any algebraic number $\alpha$ of degree $n$, the generalized algorithm produces an eventually periodic sequence. Conversely, if the sequence is eventually periodic, then the input is an algebraic number of degree at most $n$.
\end{theorem}

This establishes the equivalence of the approaches and places them within a broader theoretical context, demonstrating the robustness and completeness of the solution to Hermite's problem.
