\section{Computational Aspects of the Matrix Approach}\label{sec:matrix_computational}

Having established the matrix approach using companion matrices and trace sequences as a solution to Hermite's problem, this section focuses on its numerical validation and computational aspects.

\subsection{Numerical Validation}

Our implementation and testing demonstrate exceptional accuracy and efficiency in identifying cubic irrationals.

\begin{table}[ht]
\centering
\small
\caption{Results of Matrix Verification Method on Different Number Types}
\label{tab:matrix_results}
\begin{tabular}{|l|l|l|c|}
\hline
\textbf{Number} & \textbf{Type} & \textbf{Result} & \textbf{Correct?} \\
\hline
$\sqrt{2}$ & Quadratic & Not Cubic & \checkmark \\
$\sqrt{3}$ & Quadratic & Not Cubic & \checkmark \\
$\frac{1+\sqrt{5}}{2}$ & Quadratic & Not Cubic & \checkmark \\
\hline
$\sqrt[3]{2}$ & Cubic & Cubic & \checkmark \\
$\sqrt[3]{3}$ & Cubic & Cubic & \checkmark \\
$1+\sqrt[3]{2}$ & Cubic & Cubic & \checkmark \\
\hline
$\pi$ & Transcendental & Not Cubic & \checkmark \\
$e$ & Transcendental & Not Cubic & \checkmark \\
\hline
$\frac{3}{2}$ & Rational & Not Cubic & \checkmark \\
$\frac{22}{7}$ & Rational & Not Cubic & \checkmark \\
\hline
\end{tabular}
\end{table}

The matrix verification method achieves 100\% accuracy in our test cases, correctly identifying all cubic irrationals and properly classifying non-cubic numbers.

\begin{example}[Detailed Analysis of Cube Root of 2]
For $\alpha = 2^{1/3}$ with minimal polynomial $p(x) = x^3 - 2$:
\begin{enumerate}
    \item Companion matrix: $C = \begin{pmatrix} 0 & 0 & 2 \\ 1 & 0 & 0 \\ 0 & 1 & 0 \end{pmatrix}$
    \item Traces: $\tr(C^0) = 3$, $\tr(C^1) = 0$, $\tr(C^2) = 0$, $\tr(C^3) = 6$, $\tr(C^4) = 0$, $\tr(C^5) = 0$
    \item Verification: The trace relations hold perfectly for all $k \geq 3$:
    \begin{align*}
        \tr(C^3) &= 0 \cdot \tr(C^2) + 0 \cdot \tr(C^1) + 2 \cdot \tr(C^0) = 0 + 0 + 2 \cdot 3 = 6 \\
        \tr(C^4) &= 0 \cdot \tr(C^3) + 0 \cdot \tr(C^2) + 2 \cdot \tr(C^1) = 0 + 0 + 2 \cdot 0 = 0 \\
        \tr(C^5) &= 0 \cdot \tr(C^4) + 0 \cdot \tr(C^3) + 2 \cdot \tr(C^2) = 0 + 0 + 2 \cdot 0 = 0
    \end{align*}
\end{enumerate}
The perfect alignment of these trace relations confirms that $2^{1/3}$ is a cubic irrational.
\end{example}

\subsection{Comparison with Other Approaches}

\begin{table}[ht]
\centering
\small
\caption{Comparison of the Three Solution Approaches}
\label{tab:comparison}
\begin{tabular}{|p{0.28\textwidth}|p{0.28\textwidth}|p{0.28\textwidth}|}
\hline
\textbf{HAPD Algorithm} & \textbf{Matrix Approach} & \textbf{Modified sin² Algorithm} \\
\hline
% Strengths/Characteristics
Works from $\alpha$ directly & Needs minimal polynomial & Works from $\alpha$ directly \\
\hline
Geometric approach & Algebraic approach & Floor function based \\
\hline
Gives representation pairs & Produces trace sequence & Gives representation pairs \\
\hline
Handles complex roots well & Handles complex roots well & Adapted for complex roots \\
\hline
% Weaknesses/Considerations
Computationally slower & Requires polynomial ID & Sensitive to floor details \\
\hline
May have precision issues & Robust with polynomial & May have precision issues \\
\hline
Based on Hermite's ideas & Efficient verification & Extends Karpenkov's work \\
\hline
\end{tabular}
\end{table}

The matrix approach excels in computational efficiency and numerical stability once a candidate minimal polynomial is found. However, finding this polynomial typically requires algorithms like PSLQ or LLL, which themselves can be computationally intensive.

The \HAPD{} algorithm, in contrast, works directly with the real number without requiring prior identification of its minimal polynomial, and provides a representation system that more directly addresses Hermite's original vision. The modified sin²-algorithm offers another alternative, particularly adapted from existing methods for totally real fields.

\subsection{Implementation Strategy}

In practice, we recommend a combined approach:
\begin{enumerate}
    \item Run a few iterations of the \HAPD{} algorithm to quickly identify rational numbers and detect evidence of periodicity for cubic irrationals.
    \item For potential cubic irrationals, use PSLQ or LLL to find a candidate minimal polynomial.
    \item Confirm using the matrix verification method, which provides high accuracy with minimal computational overhead once the polynomial is identified.
\end{enumerate}

This hybrid approach leverages the strengths of multiple methods, providing a robust and efficient solution to identifying and characterizing cubic irrationals in practice.
