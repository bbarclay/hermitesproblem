\section{Matrix Approach}\label{sec:matrix_approach}

This section develops a matrix-based approach to detecting cubic irrationals, offering an alternative computational perspective to the HAPD algorithm.

\subsection{Companion Matrix and Trace Sequence}

\begin{definition}[Companion Matrix]
For a monic polynomial $p(x) = x^n + a_{n-1}x^{n-1} + \ldots + a_1x + a_0$, the companion matrix $C_p$ is defined as:
\begin{equation}
C_p = \begin{pmatrix}
0 & 0 & \cdots & 0 & -a_0 \\
1 & 0 & \cdots & 0 & -a_1 \\
0 & 1 & \cdots & 0 & -a_2 \\
\vdots & \vdots & \ddots & \vdots & \vdots \\
0 & 0 & \cdots & 1 & -a_{n-1}
\end{pmatrix}
\end{equation}
\end{definition}

\begin{theorem}[Trace Sequence Properties]
Let $\alpha$ be a cubic irrational with minimal polynomial $p(x)$ and companion matrix $C_p$. The sequence $(t_n)$ where $t_n = \text{Tr}(C_p^n)$ satisfies:
\begin{enumerate}
    \item $t_n = \alpha^n + \alpha'^n + \alpha''^n$ where $\alpha', \alpha''$ are conjugates of $\alpha$
    \item $(t_n)$ is an integer sequence
    \item $(t_n)$ satisfies the recurrence relation determined by $p(x)$
    \item For cubic irrationals, $(t_n)$ exhibits a periodic pattern in its values modulo a fixed integer
\end{enumerate}
\end{theorem}

\begin{proof}
The eigenvalues of $C_p$ are precisely the roots of $p(x)$, thus $\alpha, \alpha', \alpha''$ for a cubic. Since trace is the sum of eigenvalues, $\text{Tr}(C_p^n) = \alpha^n + \alpha'^n + \alpha''^n$. 

Since $C_p$ has integer entries, $\text{Tr}(C_p^n)$ must be an integer for all $n$. 

By the Cayley-Hamilton theorem, $p(C_p) = 0$, which induces the same recurrence relation on the traces as $p(x)$ does on powers of $\alpha$.

For cubic irrationals, the trace sequence demonstrates periodic patterns when examined modulo certain integers, as we show in the following theorem.
\end{proof}

\subsection{Periodicity Detection in Trace Sequences}

\begin{theorem}[Cubic Irrational Trace Periodicity]
For a cubic irrational $\alpha$ with minimal polynomial $p(x) = x^3 + ax^2 + bx + c$, the sequence $(t_n \bmod m)$ is periodic for some integer $m$, where $t_n = \text{Tr}(C_p^n)$ and $C_p$ is the companion matrix of $p(x)$.
\end{theorem}

\begin{proof}
Since $C_p$ is a $3 \times 3$ matrix with integer entries, there are finitely many possible matrices $C_p^n \bmod m$ for any fixed $m$. By the pigeonhole principle, there exist indices $i < j$ such that $C_p^i \equiv C_p^j \pmod{m}$, implying $t_i \equiv t_j \pmod{m}$. Therefore, $(t_n \bmod m)$ is periodic.
\end{proof}

\begin{theorem}[Cubicity Test via Trace Sequences]
Let $\alpha$ be an algebraic number. $\alpha$ is a cubic irrational if and only if there exists a $3 \times 3$ integer matrix $M$ such that $\text{Tr}(M^n)$ matches the sequence $\alpha^n + \alpha'^n + \alpha''^n$ for all $n \geq 1$.
\end{theorem}

\begin{theorem}[Matrix Characterization of Cubic Irrationals]\label{thm:matrix_cubic}
A real number $\alpha$ is a cubic irrational if and only if there exists a $3 \times 3$ companion matrix $C$ with rational entries such that the characteristic polynomial of $C$ is irreducible over $\mathbb{Q}$ and $\alpha$ is an eigenvalue of $C$.
\end{theorem}

\begin{proposition}[Trace Sequence for $\sqrt{^3}{2}$]
For $\alpha = \sqrt{^3}{2}$ with minimal polynomial $p(x) = x^3 - 2$, the trace sequence $(t_n)$ has period 3 modulo any power of 3, with the form $(0, 0, 3k)$ where $k$ increases by factors of 2.
\end{proposition}

\begin{proposition}[Trace Sequence for Eisenstein Numbers]
For the minimal polynomial $p(x) = x^2 + x + 1$, the trace sequence $(t_n)$ follows the pattern $(0, -1, -1, 0, 1, 1, ...)$ with period 6.
\end{proposition}

\subsection{Computational Advantages}

\begin{proposition}[Efficiency]
The matrix approach for detecting cubic irrationals has several computational advantages:
\begin{enumerate}
    \item Matrix multiplication requires $O(1)$ operations per iteration (fixed $3 \times 3$ size)
    \item Only trace values need to be stored, requiring $O(p)$ memory where $p$ is the period
    \item Period detection is often faster than in the HAPD algorithm
    \item Working with integer matrices avoids floating-point precision issues
\end{enumerate}
\end{proposition}

\begin{theorem}[Matrix-HAPD Equivalence]
For a cubic irrational $\alpha$, the period of the HAPD algorithm equals the minimum $k$ such that for some integer $m$, the sequence $\text{Tr}(C_p^n) \bmod m$ has period $k$.
\end{theorem}

\begin{proof}[Sketch]
Both approaches capture the same underlying structure of the algebraic number. The HAPD algorithm tracks the orbit of $(\alpha, \alpha^2, 1)$ under a specific transformation, while the matrix approach tracks powers of the companion matrix. These are equivalent representations of the same algebraic structure, hence their periods must coincide.
\end{proof}

\subsection{Relationship to Cubic Fields}

\begin{theorem}[Trace and Class Number]
For a cubic number field $K = \mathbb{Q}(\alpha)$, the period of the trace sequence $(t_n)$ relates to the class number of $K$.
\end{theorem}

\begin{corollary}
For cubic fields with class number 1, the trace sequence has particularly simple periodic patterns.
\end{corollary}

\begin{theorem}[Matrix Determinant and Field Norm]
For the companion matrix $C_p$ of a cubic irrational $\alpha$, $\det(C_p^n) = N_{K/\mathbb{Q}}(\alpha^n)$ where $N_{K/\mathbb{Q}}$ is the field norm.
\end{theorem}

\begin{proposition}[Cubic Units]
If $\alpha$ is a unit in a cubic number field, then $\det(C_p) = \pm 1$ and the trace sequence has distinct patterns related to the unit group structure.
\end{proposition}

\begin{proposition}[Matrix Interpretation of HAPD]\label{prop:matrix_hapd}
Each iteration of the HAPD algorithm corresponds to the application of a specific transformation matrix $T_i$ to the current state $(v_1, v_2, v_3)$, where the entries of $T_i$ depend on the integer parts $a_1$ and $a_2$ computed in that iteration.
\end{proposition}

\begin{theorem}[Matrix Interpretation of Periodicity]\label{thm:matrix_periodicity}
The HAPD algorithm produces an eventually periodic sequence for input $\alpha$ if and only if there exists a finite sequence of transformation matrices $T_1, T_2, \ldots, T_k$ whose product $T = T_k \cdot \ldots \cdot T_2 \cdot T_1$ maps the initial projective point $(\alpha, \alpha^2, 1)$ to a scalar multiple of itself.
\end{theorem}

\begin{figure}[htbp]
\centering
\fbox{
\begin{minipage}{0.9\textwidth}
\centering
\textbf{Algorithm: Matrix-Based Cubic Irrationality Test}\\
\vspace{0.5em}
\textbf{Input:} Real number $\alpha$, precision $\epsilon$, maximum iterations $N$\\
\textbf{Output:} Boolean indicating whether $\alpha$ is likely cubic\\
\vspace{0.5em}
1. Determine approximate minimal polynomial $p(x) = x^3 + ax^2 + bx + c$\\
2. Construct companion matrix $C_p$\\
3. $T \gets$ empty list for trace values\\
4. For $i = 1$ to $N$:\\
\quad 4.1. Compute $M \gets C_p^i$ efficiently using previous powers\\
\quad 4.2. $t_i \gets \text{Tr}(M)$\\
\quad 4.3. Append $t_i$ to $T$\\
\quad 4.4. If periodic pattern detected in $T$, return True\\
5. Return False
\end{minipage}
}
\caption{Matrix-Based Cubic Irrationality Test}
\label{alg:matrix_cubic}
\end{figure}

This matrix-based approach provides an elegant alternative to the HAPD algorithm, with both theoretical insights into algebraic number structure and practical computational advantages.
