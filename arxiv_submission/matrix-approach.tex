\section{Matrix Approach}\label{sec:matrix_approach}

The matrix approach offers a direct method for detecting cubic irrationals with distinct computational advantages.

\subsection{Companion Matrix and Trace Sequence}

\begin{definition}[Companion Matrix \cite{Horn2012}]
For a monic polynomial $p(x) = x^n + a_{n-1}x^{n-1} + \ldots + a_1x + a_0$, the companion matrix $C_p$ is defined as:
\begin{equation}
C_p = \begin{pmatrix}
0 & 0 & \cdots & 0 & -a_0 \\
1 & 0 & \cdots & 0 & -a_1 \\
0 & 1 & \cdots & 0 & -a_2 \\
\vdots & \vdots & \ddots & \vdots & \vdots \\
0 & 0 & \cdots & 1 & -a_{n-1}
\end{pmatrix}
\end{equation}
\end{definition}

\begin{theorem}[Trace Sequence Properties \cite{Horn2012}]
Let $\alpha$ be a cubic irrational with minimal polynomial $p(x)$ and companion matrix $C_p$. The sequence $(t_n)$ where $t_n = \text{Tr}(C_p^n)$ satisfies:
\begin{enumerate}
    \item $t_n = \alpha^n + \alpha'^n + \alpha''^n$ where $\alpha', \alpha''$ are conjugates of $\alpha$
    \item $(t_n)$ is an integer sequence
    \item $(t_n)$ satisfies the recurrence relation determined by $p(x)$
    \item For cubic irrationals, $(t_n)$ exhibits periodic patterns modulo a fixed integer
\end{enumerate}
\end{theorem}

\begin{proof}
The eigenvalues of $C_p$ are the roots of $p(x)$: $\alpha, \alpha', \alpha''$. Since trace is the sum of eigenvalues, $\text{Tr}(C_p^n) = \alpha^n + \alpha'^n + \alpha''^n$. 

$C_p$ has integer entries, so $\text{Tr}(C_p^n)$ must be an integer for all $n$. 

By the Cayley-Hamilton theorem, $p(C_p) = 0$, inducing the same recurrence relation on the traces as $p(x)$ does on powers of $\alpha$.

The trace sequence demonstrates periodic patterns when examined modulo certain integers, as shown in the following theorem.
\end{proof}

\subsection{Periodicity Detection in Trace Sequences}

\begin{theorem}[Cubic Irrational Trace Periodicity \cite{Cohen1993}]
For a cubic irrational $\alpha$ with minimal polynomial $p(x) = x^3 + ax^2 + bx + c$, the sequence $(t_n \bmod m)$ is periodic for some integer $m$, where $t_n = \text{Tr}(C_p^n)$ and $C_p$ is the companion matrix of $p(x)$.
\end{theorem}

\begin{proof}
Since $C_p$ is a $3 \times 3$ matrix with integer entries, there are finitely many possible matrices $C_p^n \bmod m$ for any fixed $m$. By the pigeonhole principle, there exist indices $i < j$ such that $C_p^i \equiv C_p^j \pmod{m}$, implying $t_i \equiv t_j \pmod{m}$. Therefore, $(t_n \bmod m)$ is periodic.
\end{proof}

\begin{theorem}[Cubicity Test via Trace Sequences]
Let $\alpha$ be an algebraic number. $\alpha$ is a cubic irrational if and only if there exists a $3 \times 3$ integer matrix $M$ such that $\text{Tr}(M^n)$ matches the sequence $\alpha^n + \alpha'^n + \alpha''^n$ for all $n \geq 1$.
\end{theorem}

\begin{theorem}[Matrix Characterization of Cubic Irrationals]\label{thm:matrix_cubic}
A real number $\alpha$ is a cubic irrational if and only if there exists a $3 \times 3$ companion matrix $C$ with rational entries such that the characteristic polynomial of $C$ is irreducible over $\mathbb{Q}$ and $\alpha$ is an eigenvalue of $C$.
\end{theorem}

\begin{proposition}[Trace Sequence for $\sqrt{^3}{2}$]
For $\alpha = \sqrt{^3}{2}$ with minimal polynomial $p(x) = x^3 - 2$, the trace sequence $(t_n)$, starting with $t_0=3$, has the structure $t_k = 0$ if $k \not\equiv 0 \pmod{3}$. For terms where $k = 3j$ for $j \geq 1$, the sequence is $t_{3j} = 3 \cdot 2^j$. Consequently, when taken modulo $3^p$ for $p \ge 1$, the sequence $(t_n \pmod{3^p})$ is periodic.
\end{proposition}

\begin{proposition}[Trace Sequence for Eisenstein Numbers \cite{Cox2012}]
For the minimal polynomial $p(x) = x^2 + x + 1$, the trace sequence $(t_n)$ follows the pattern $(0, -1, -1, 0, 1, 1, ...)$ with period 6.
\end{proposition}

\subsection{The Matrix Verification Method}

The matrix verification method directly determines whether a number $\alpha$ is a cubic irrational by analyzing properties of its associated companion matrix.

\begin{algorithm}
\caption{Matrix-Based Cubic Irrational Detection}
\label{alg:matrix_verification}
\begin{algorithmic}[1]
\Procedure{MatrixVerifyCubic}{$\alpha$, tolerance}
    \State Find candidate minimal polynomial $p(x) = x^3 + ax^2 + bx + c$
    \State Create companion matrix $C = \begin{pmatrix} 0 & 0 & -c \\ 1 & 0 & -b \\ 0 & 1 & -a \end{pmatrix}$
    
    \State Compute powers $C^k$ for $k = 0, 1, 2, 3, 4, 5$
    \State Compute traces $\tr(C^k)$ for each power
    
    \State Verify trace relations:
    \For{$k = 3, 4, 5$}
        \State $\text{expected}_k \gets -a \cdot \tr(C^{k-1}) - b \cdot \tr(C^{k-2}) - c \cdot \tr(C^{k-3})$ % Corrected recurrence relation sign based on Thm in matrix-verification
        \If{$|\tr(C^k) - \text{expected}_k| > \text{tolerance}$}
            \State \Return "Not a cubic irrational"
        \EndIf
    \EndFor
    
    \State \Return "Confirmed cubic irrational with minimal polynomial $p(x)$"
\EndProcedure
\end{algorithmic}
\end{algorithm}

\subsection{Theoretical Foundation via Trace Relations}

\begin{theorem}[Trace Relations for Cubic Irrationals]\label{thm:trace_relations}
Let $\alpha$ be a cubic irrational with minimal polynomial $p(x) = x^3 + ax^2 + bx + c$, and let $C$ be the companion matrix of $p(x)$. Then for all $k \geq 3$:
\begin{equation}
\tr(C^k) = -a \cdot \tr(C^{k-1}) - b \cdot \tr(C^{k-2}) - c \cdot \tr(C^{k-3})
\end{equation}
with initial conditions $\tr(C^0) = 3$, $\tr(C^1) = -a$, and $\tr(C^2) = a^2-2b$. % Corrected initial conditions
\end{theorem}

\begin{proof}
The companion matrix $C$ has characteristic polynomial $p(x) = x^3 + ax^2 + bx + c$, and its eigenvalues are the roots of $p(x)$: $\alpha, \beta, \gamma$.

For any $k \geq 0$, $\tr(C^k) = \alpha^k + \beta^k + \gamma^k$, the sum of the $k$-th powers of the roots, denoted $s_k$.

From the minimal polynomial (Newton's sums), we know the recurrence relation:
\begin{equation}
s_k = -a \cdot s_{k-1} - b \cdot s_{k-2} - c \cdot s_{k-3} \quad \text{for } k \geq 3
\end{equation}
The initial conditions are $s_0 = 3$, $s_1 = -a$, $s_2 = a^2 - 2b$. 
Since $s_k = \tr(C^k)$, the theorem follows.
\end{proof}

\begin{corollary}[Matrix Characterization via Trace Relations]\label{cor:matrix_characterization_trace}
A real number $\alpha$ is a cubic irrational if and only if there exists a monic irreducible cubic polynomial $p(x) = x^3 + ax^2 + bx + c$ such that $p(\alpha) = 0$ and the companion matrix $C$ of $p(x)$ satisfies the trace relations in Theorem \ref{thm:trace_relations}.
\end{corollary}

\begin{proof}
This follows directly from Theorem \ref{thm:trace_relations} and the definition of a cubic irrational as a root of an irreducible cubic polynomial with rational coefficients.
\end{proof}

\begin{example}[Detailed Verification for Cube Root of 2]
For $\alpha = 2^{1/3}$ with minimal polynomial $p(x) = x^3 - 2$ (so $a=0, b=0, c=-2$):
\begin{enumerate}
    \item Companion matrix: $C = \begin{pmatrix} 0 & 0 & 2 \\ 1 & 0 & 0 \\ 0 & 1 & 0 \end{pmatrix}$
    \item Initial Traces: $\tr(C^0) = 3$, $\tr(C^1) = -a = 0$, $\tr(C^2) = a^2-2b = 0$. 
    \item Higher Traces: $\tr(C^3) = 6$, $\tr(C^4) = 0$, $\tr(C^5) = 0$.
    \item Verification using $k=3$: $\tr(C^3) = -a \cdot \tr(C^2) - b \cdot \tr(C^1) - c \cdot \tr(C^0) = -0(0) - 0(0) - (-2)(3) = 6$. Matches.
    \item Verification using $k=4$: $\tr(C^4) = -a \cdot \tr(C^3) - b \cdot \tr(C^2) - c \cdot \tr(C^1) = -0(6) - 0(0) - (-2)(0) = 0$. Matches.
    \item Verification using $k=5$: $\tr(C^5) = -a \cdot \tr(C^4) - b \cdot \tr(C^3) - c \cdot \tr(C^2) = -0(0) - 0(6) - (-2)(0) = 0$. Matches.
\end{enumerate}
The perfect alignment of these trace relations confirms that $2^{1/3}$ is a cubic irrational.
\end{example}

\subsection{Computational Advantages}

\begin{proposition}[Efficiency \cite{Ferguson1999}]
The matrix approach offers several computational advantages:
\begin{enumerate}
    \item Fixed $3 \times 3$ matrix size requires O(1) operations per iteration
    \item Storage limited to trace values: O(p) memory where p is the period
    \item Typically faster period detection than HAPD algorithm
    \item Integer matrices avoid floating-point precision issues
\end{enumerate}
\end{proposition}

\begin{theorem}[Matrix-HAPD Equivalence]
For a cubic irrational $\alpha$, the period of the HAPD algorithm equals the minimum $k$ such that for some integer $m$, the sequence $\text{Tr}(C_p^n) \bmod m$ has period $k$.
\end{theorem}

\begin{proof}[Sketch]
Both approaches capture the same underlying structure. The HAPD algorithm tracks the orbit of $(\alpha, \alpha^2, 1)$ under a specific transformation, while the matrix approach tracks powers of the companion matrix. These represent the same algebraic structure, hence their periods coincide.
\end{proof}

\subsection{Relationship to Cubic Fields}

\begin{theorem}[Trace and Class Number \cite{Cohen1993}]
For a cubic number field $K = \mathbb{Q}(\alpha)$, the period of the trace sequence $(t_n)$ relates to the class number of $K$.
\end{theorem}

\begin{corollary}
For cubic fields with class number 1, the trace sequence has particularly simple periodic patterns.
\end{corollary}

\begin{theorem}[Matrix Determinant and Field Norm \cite{Cox2012}]
For the companion matrix $C_p$ of a cubic irrational $\alpha$, $\det(C_p^n) = N_{K/\mathbb{Q}}(\alpha^n)$ where $N_{K/\mathbb{Q}}$ is the field norm.
\end{theorem}

\begin{proposition}[Cubic Units \cite{Cohen1993}]
If $\alpha$ is a unit in a cubic number field, then $\det(C_p) = \pm 1$ and the trace sequence has distinct patterns related to the unit group structure.
\end{proposition}

\begin{proposition}[Matrix Interpretation of HAPD]\label{prop:matrix_hapd}
Each HAPD iteration corresponds to applying a transformation matrix $T_i$ to the current state $(v_1, v_2, v_3)$, where $T_i$ entries depend on the integer parts $a_1$ and $a_2$ computed in that iteration.
\end{proposition}

\begin{theorem}[Matrix Interpretation of Periodicity]\label{thm:matrix_periodicity}
The HAPD algorithm produces an eventually periodic sequence for input $\alpha$ if and only if there exists a finite sequence of transformation matrices $T_1, T_2, \ldots, T_k$ whose product $T = T_k \cdot \ldots \cdot T_2 \cdot T_1$ maps the initial projective point $(\alpha, \alpha^2, 1)$ to a scalar multiple of itself.
\end{theorem}

\begin{figure}[htbp]
\centering
\fbox{
\begin{minipage}{0.9\textwidth}
\centering
\textbf{Algorithm: Matrix-Based Cubic Irrationality Test}\\
\vspace{0.5em}
\textbf{Input:} Real number $\alpha$, precision $\epsilon$, maximum iterations $N$\\
\textbf{Output:} Boolean indicating whether $\alpha$ is likely cubic\\
\vspace{0.5em}
1. Determine approximate minimal polynomial $p(x) = x^3 + ax^2 + bx + c$\\
2. Construct companion matrix $C_p$\\
3. $T \gets$ empty list for trace values\\
4. For $i = 1$ to $N$:\\
\quad 4.1. Compute $M \gets C_p^i$ efficiently using previous powers\\
\quad 4.2. $t_i \gets \text{Tr}(M)$\\
\quad 4.3. Append $t_i$ to $T$\\
\quad 4.4. If periodic pattern detected in $T$, return True\\
5. Return False
\end{minipage}
}
\caption{Matrix-Based Cubic Irrationality Test}
\label{alg:matrix_cubic}
\end{figure}

This matrix approach provides both theoretical insights into algebraic number structure and practical computational advantages \cite{Ferguson1999, Lenstra1982}.
