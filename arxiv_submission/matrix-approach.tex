\section{Matrix Approach and Verification}\label{sec:matrix_approach}

We present a unified matrix-based framework for detecting and verifying cubic irrationals, combining theoretical foundations with practical computational methods.

\subsection{Companion Matrix Theory}

\begin{definition}[Companion Matrix]\label{def:companion_matrix}
For a monic polynomial $p(x) = x^n + a_{n-1}x^{n-1} + \ldots + a_1x + a_0$, the companion matrix $C_p$ is defined as:
\begin{equation}
C_p = \begin{pmatrix}
0 & 0 & \cdots & 0 & -a_0 \\
1 & 0 & \cdots & 0 & -a_1 \\
0 & 1 & \cdots & 0 & -a_2 \\
\vdots & \vdots & \ddots & \vdots & \vdots \\
0 & 0 & \cdots & 1 & -a_{n-1}
\end{pmatrix}
\end{equation}
\end{definition}

\begin{theorem}[Trace Sequence Properties]\label{thm:trace_properties}
Let $\alpha$ be a cubic irrational with minimal polynomial $p(x) = x^3 + ax^2 + bx + c$ and companion matrix $C_p$. The sequence $(t_n)_{n \geq 0}$ where $t_n = \tr(C_p^n)$ satisfies:
\begin{enumerate}
    \item $t_n = \alpha^n + (\alpha')^n + (\alpha'')^n$ where $\alpha', \alpha''$ are conjugates of $\alpha$
    \item $(t_n)_{n \geq 0}$ is an integer sequence
    \item $(t_n)_{n \geq 0}$ satisfies the linear recurrence relation determined by $p(x)$
    \item For cubic irrationals, $(t_n \bmod m)_{n \geq 0}$ is periodic for some integer $m > 1$
\end{enumerate}
\end{theorem}

\begin{proof}
The eigenvalues of $C_p$ are precisely the roots of $p(x)$: $\alpha, \alpha', \alpha''$. Since trace is the sum of eigenvalues, $\tr(C_p^n) = \alpha^n + (\alpha')^n + (\alpha'')^n$.

$C_p$ has integer entries, so $\tr(C_p^n)$ must be an integer for all $n \geq 0$.

By the Cayley-Hamilton theorem, $p(C_p) = 0$, which induces a recurrence relation on the traces identical to that satisfied by the power sums of the roots of $p(x)$.

For the periodicity modulo $m$, note that there are only finitely many possible $3 \times 3$ matrices with integer entries modulo $m$. By the pigeonhole principle, the sequence of powers $(C_p^n \bmod m)_{n \geq 0}$ must eventually repeat, forcing the trace sequence to be periodic modulo $m$ as well.
\end{proof}

\subsection{Trace Relations and Verification}

\begin{theorem}[Trace Relations for Cubic Irrationals]\label{thm:trace_relations}
Let $\alpha$ be a cubic irrational with minimal polynomial $p(x) = x^3 + ax^2 + bx + c$, and let $C_p$ be the companion matrix of $p(x)$. Then for all $k \geq 3$:
\begin{equation}
\tr(C_p^k) = -a \cdot \tr(C_p^{k-1}) - b \cdot \tr(C_p^{k-2}) - c \cdot \tr(C_p^{k-3})
\end{equation}
with initial conditions $\tr(C_p^0) = 3$, $\tr(C_p^1) = -a$, and $\tr(C_p^2) = a^2-2b$.
\end{theorem}

\begin{proof}
The companion matrix $C_p$ has characteristic polynomial $p(x) = x^3 + ax^2 + bx + c$, and its eigenvalues are the roots of $p(x)$: $\alpha, \alpha', \alpha''$.

For any $k \geq 0$, $\tr(C_p^k) = \alpha^k + (\alpha')^k + (\alpha'')^k$, the sum of the $k$-th powers of the roots, denoted $s_k$.

From Newton's identities relating coefficients and power sums:
\begin{equation}
s_k = -a \cdot s_{k-1} - b \cdot s_{k-2} - c \cdot s_{k-3} \quad \text{for } k \geq 3
\end{equation}

The initial conditions follow from:
\begin{align}
\tr(C_p^0) &= \tr(I) = 3 \\
\tr(C_p^1) &= \tr(C_p) = -a \\
\tr(C_p^2) &= \tr(C_p \cdot C_p) = a^2-2b
\end{align}
\end{proof}

\subsection{Verification Algorithm}

\begin{algorithm}[H]
\caption{Matrix-Based Cubic Irrational Verification}
\label{alg:matrix_verification}
\begin{algorithmic}[1]
\Procedure{MatrixVerifyCubic}{$\alpha$, tolerance}
    \State Find candidate minimal polynomial $p(x) = x^3 + ax^2 + bx + c$
    \State Create companion matrix $C_p = \begin{pmatrix} 0 & 0 & -c \\ 1 & 0 & -b \\ 0 & 1 & -a \end{pmatrix}$
    \State Compute powers $(C_p^k)_{k=0}^5$
    \State Compute traces $\tr(C_p^k)$ for each power
    \For{$k = 3, 4, 5$}
        \State $\text{expected}_k \gets -a \cdot \tr(C_p^{k-1}) - b \cdot \tr(C_p^{k-2}) - c \cdot \tr(C_p^{k-3})$
        \If{$|\tr(C_p^k) - \text{expected}_k| > \text{tolerance}$}
            \State \Return "Not a cubic irrational"
        \EndIf
    \EndFor
    \State \Return "Confirmed cubic irrational with minimal polynomial $p(x)$"
\EndProcedure
\end{algorithmic}
\end{algorithm}

\subsection{Numerical Validation}

Our implementation demonstrates exceptional accuracy in identifying cubic irrationals:

\begin{table}[htbp]
\centering
\begin{tabular}{|l|c|c|c|}
\hline
\textbf{Number Type} & \textbf{Example} & \textbf{Candidate Polynomial} & \textbf{Verified?} \\
\hline
Rational & $\frac{22}{7}$ & $x - \frac{22}{7}$ & Yes (degree 1) \\
\hline
Quadratic Irrational & $\sqrt{2}$ & $x^2 - 2$ & Yes (degree 2) \\
\hline
Cubic Irrational & $\sqrt[3]{2}$ & $x^3 - 2$ & Yes (degree 3) \\
\hline
Cubic (Complex Conj.) & $\sqrt[3]{2} + 0.1$ & $x^3 - 0.3x^2 - 0.03x - 2.003$ & Yes (degree 3) \\
\hline
Transcendental & $\pi$ & Various approximations & No \\
\hline
\end{tabular}
\caption{Results of Matrix Verification Method on Different Number Types}
\label{tab:matrix_verification_examples}
\end{table}

\subsection{Comparison with Other Methods}

\begin{table}[htbp]
\centering
\begin{tabular}{|l|c|c|c|}
\hline
\textbf{Feature} & \textbf{\HAPD{} Algorithm} & \textbf{Matrix Approach} & \textbf{Subtractive Algorithm} \\
\hline
Prior knowledge & None & Minimal polynomial & None \\
\hline
Computational complexity & $O(M^3)$ iters & $O(1)$ matrix ops & $O(M^2)$ iters \\
\hline
Geometric interpretation & Clear & Limited & Clear \\
\hline
Algebraic interpretation & Limited & Clear algebraic interpretation & Moderate \\
\hline
Implementation difficulty & Moderate & Easy & Easy \\
\hline
Numerical stability & Sensitive & Robust & Very robust \\
\hline
Sensitivity to phase-shifts & High & None & Medium \\
\hline
Detects rational/quadratic & Yes (terminates/aperiodic) & Yes (verified degree) & Yes (terminates) \\
\hline
Extended to complex case & Yes, with care & Robust once polynomial is known & Yes, straightforwardly \\
\hline
\end{tabular}
\caption{Comparison of the Three Solution Approaches}
\label{tab:verification_comparison}
\end{table}

\subsection{Implementation Strategy}

For practical applications, we recommend a combined approach:
\begin{enumerate}
    \item Run a few iterations of the \HAPD{} algorithm to quickly identify rational numbers and detect evidence of periodicity for cubic irrationals.
    \item For potential cubic irrationals, use PSLQ or LLL to find a candidate minimal polynomial.
    \item Confirm using the matrix verification method, which provides high accuracy with minimal computational overhead once the polynomial is identified.
\end{enumerate}

This hybrid approach leverages the strengths of multiple methods while mitigating their individual weaknesses.
