\section{Galois Theoretic Proof of Non-Periodicity}\label{sec:galois_theory}

Cubic irrationals cannot have periodic continued fraction expansions, necessitating our higher-dimensional approach.

\begin{definition}[Continued Fraction Expansion]
For $\alpha \in \mathbb{R}$, the continued fraction expansion is $[a_0; a_1, a_2, \ldots]$ where $a_0 = \floor{\alpha}$ and for $i \geq 1$, $a_i = \floor{\alpha_i}$ with $\alpha_0 = \alpha$ and $\alpha_{i+1} = \frac{1}{\alpha_i - a_i}$.
\end{definition}

\begin{definition}[Eventually Periodic Continued Fraction]
A continued fraction $[a_0; a_1, a_2, \ldots]$ is eventually periodic if $\exists N \geq 0, p > 0$ such that $a_{N+i} = a_{N+p+i}$ for all $i \geq 0$, denoted as 
\begin{equation}
[a_0; a_1, \ldots, a_{N-1}, \overline{a_N, \ldots, a_{N+p-1}}]
\end{equation}
\end{definition}

\begin{theorem}[Lagrange \cite{Lagrange1770}]\label{thm:lagrange}
A real number has an eventually periodic continued fraction expansion if and only if it is a quadratic irrational.
\end{theorem}

\begin{definition}[Minimal Polynomial]
For an algebraic number $\alpha$ over $\Q$, the minimal polynomial of $\alpha$ over $\Q$ is the monic polynomial $\text{min}_\Q(\alpha, x) \in \Q[x]$ of least degree such that $\text{min}_\Q(\alpha, x)(\alpha) = 0$.
\end{definition}

\begin{definition}[Cubic Irrational]
A real number $\alpha$ is a cubic irrational if it is a root of an irreducible polynomial of degree 3 with rational coefficients.
\end{definition}

\begin{definition}[Galois Group \cite{Cox2012}]
Let $L/K$ be a field extension. If $L$ is the splitting field of a 
separable polynomial over $K$, then $\text{Aut}_K(L)$ is the Galois group 
of $L$ over $K$, denoted $\Gal(L/K)$.
\end{definition}

\begin{theorem}[Galois Groups of Cubic Polynomials \cite{Cox2012}]\label{thm:cubic_galois}
For an irreducible cubic polynomial $f(x) = x^3 + px^2 + qx + r \in \Q[x]$, the Galois group $\Gal(L/\Q)$, where $L$ is the splitting field of $f$, is isomorphic to either:
\begin{enumerate}
    \item $S_3$ if the discriminant $\Delta = -4p^3r + p^2q^2 - 4q^3 - 27r^2 + 18pqr$ is not a perfect square in $\Q$;
    \item $C_3$ if the discriminant is a non-zero perfect square in $\Q$.
\end{enumerate}
\end{theorem}

\begin{proposition}\label{prop:no_intermediate_field}
For an irreducible cubic polynomial with Galois group $S_3$ or $C_3$, there is no intermediate field between $\Q$ and $\Q(\alpha)$ where $\alpha$ is a root of the polynomial.
\end{proposition}

\begin{proof}
If $\Q \subset F \subset \Q(\alpha)$, then $[\Q(\alpha):\Q] = [\Q(\alpha):F] \cdot [F:\Q]$. Since $[\Q(\alpha):\Q] = 3$ and 3 is prime, either $[F:\Q] = 1$ or $[\Q(\alpha):F] = 1$, implying $F = \Q$ or $F = \Q(\alpha)$, contradicting the existence of a proper intermediate field.
\end{proof}

\begin{theorem}[Non-Periodicity of Cubic Irrationals \cite{Davenport1961}]\label{thm:non_periodicity}
Cubic irrationals cannot have eventually periodic continued fraction expansions.
\end{theorem}

\begin{proof}[Proof Outline]
The proof proceeds by contradiction in three key steps:

1. Assume $\alpha$ is a cubic irrational with minimal polynomial $f(x) = x^3 + px^2 + qx + r$ and periodic continued fraction expansion.

2. By Lagrange's theorem, $\alpha$ must satisfy a quadratic equation $A\alpha^2 + B\alpha + C = 0$ with $A, B, C \in \Z$.

3. The existence of both cubic and quadratic equations leads to a contradiction:
   \begin{itemize}
   \item Substituting the quadratic into the cubic yields a linear relation
   \item This implies the existence of a quadratic subfield $\Q(\alpha^2)$
   \item But no such intermediate field can exist by Proposition \ref{prop:no_intermediate_field}
   \end{itemize}

\end{proof}

\begin{proof}[Detailed Algebraic Verification]
Let $\alpha$ satisfy both:
\begin{align}
\alpha^3 + p\alpha^2 + q\alpha + r &= 0 \label{eq:cubic} \\
A\alpha^2 + B\alpha + C &= 0 \label{eq:quadratic}
\end{align}

From \eqref{eq:quadratic}, we have $\alpha^2 = \frac{-B\alpha - C}{A}$. Substituting this into \eqref{eq:cubic} and simplifying:

\begin{equation}
(B^2 - AC - pAB + qA^2)\alpha + (BC - pAC + rA^2) = 0
\end{equation}

For this to be satisfied, both coefficients must vanish:
\begin{align}
B^2 - AC - pAB + qA^2 &= 0 \label{eq:coeff1} \\
BC - pAC + rA^2 &= 0 \label{eq:coeff2}
\end{align}

These equations establish the existence of a quadratic subfield $\Q(\alpha^2)$, contradicting Proposition \ref{prop:no_intermediate_field}.
\end{proof}

\begin{corollary}\label{cor:cf_insufficient}
No direct generalization of continued fractions preserving the connection between periodicity and algebraic degree can characterize cubic irrationals.
\end{corollary}

The \HAPD{} algorithm, operating in three-dimensional projective space, characterizes cubic irrationals through periodicity, addressing the limitations established by \cite{Khinchin1964} and \cite{Cassels1957}.
