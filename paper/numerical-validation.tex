\section{Numerical Validation and Implementation}\label{sec:numerical_validation}

In this section, we provide numerical validation of our theoretical results through concrete implementations of both the \HAPD{} algorithm and the matrix-based approach. We present empirical evidence confirming that our methods correctly distinguish cubic irrationals from other number types and analyze the practical challenges of implementation.

\subsection{Implementation of the HAPD Algorithm}

We begin with a detailed implementation of the \HAPD{} algorithm, addressing precision requirements and numerical stability considerations.

\begin{algorithm}
\caption{Implementation of the \HAPD{} Algorithm}
\label{alg:hapd_implementation}
\begin{algorithmic}[1]
\Procedure{HAPD}{$\alpha$, max\_iterations, tolerance}
    \State $v_1 \gets \alpha$
    \State $v_2 \gets \alpha^2$
    \State $v_3 \gets 1$
    \State triples $\gets$ empty list
    \State pairs $\gets$ empty list
    
    \For{$i \gets 1$ to max\_iterations}
        \State $a_1 \gets \lfloor v_1/v_3 \rfloor$
        \State $a_2 \gets \lfloor v_2/v_3 \rfloor$
        \State $r_1 \gets v_1 - a_1 \cdot v_3$
        \State $r_2 \gets v_2 - a_2 \cdot v_3$
        \State $v_3^{\text{new}} \gets v_3 - a_1 \cdot r_1 - a_2 \cdot r_2$
        
        \State pairs.append($(a_1, a_2)$)
        
        \If{$|v_3^{\text{new}}| < \text{tolerance}$}
            \State \Return pairs, "Terminated (likely rational)"
        \EndIf
        
        \State $v_1 \gets r_1$
        \State $v_2 \gets r_2$
        \State $v_3 \gets v_3^{\text{new}}$
        
        \State $\text{triple} \gets (v_1, v_2, v_3)$
        \State Normalize triple to have norm 1
        
        \For{$j \gets 0$ to $\text{triples.length} - 1$}
            \If{$\text{ProjectivelyEquivalent(triple, triples[j], tolerance)}$}
                \State \Return pairs, "Periodic with preperiod $j$ and period $i-j$"
            \EndIf
        \EndFor
        
        \State triples.append(triple)
    \EndFor
    
    \State \Return pairs, "No periodicity detected within max\_iterations"
\EndProcedure

\Function{ProjectivelyEquivalent}{triple1, triple2, tolerance}
    \State Normalize both triples to have norm 1
    \State $\text{dotProduct} \gets \sum_{i=1}^3 \text{triple1}[i] \cdot \text{triple2}[i]$
    \State \Return $||\text{dotProduct}| - 1| < \text{tolerance}$
\EndFunction
\end{algorithmic}
\end{algorithm}

\begin{remark}
The algorithm includes normalization of each triple to unit length to improve numerical stability when comparing projective points. The function \textsc{ProjectivelyEquivalent} checks if two normalized triples represent the same point in projective space, allowing for a small numerical tolerance.
\end{remark}

\begin{proposition}[Numerical Precision Requirements]\label{prop:numerical_precision}
For reliable detection of periodicity in the \HAPD{} algorithm for a cubic irrational with minimal polynomial coefficients bounded by $M$:
\begin{enumerate}
    \item Floating-point precision of at least $O(\log M)$ bits is required
    \item The comparison tolerance should be set to approximately $2^{-p/2}$, where $p$ is the number of bits of precision
\end{enumerate}
\end{proposition}

\begin{proof}
The algorithm involves computing ratios and remainders in each iteration. For a cubic irrational with coefficients bounded by $M$, the entries in the transformation matrices are also bounded by polynomials in $M$.

Over the course of $O(M^3)$ iterations needed to detect periodicity, numerical errors can accumulate, potentially leading to false positives or negatives in periodicity detection. With $p$ bits of precision, the maximum attainable accuracy is approximately $2^{-p}$.

When comparing projective points, we compute the dot product of normalized vectors, which should be exactly 1 for identical points or -1 for antipodal points. Allowing for numerical errors, the tolerance should be on the order of $2^{-p/2}$ to account for error accumulation while still distinguishing truly distinct points.
\end{proof}

\subsection{Test Cases and Results}

We now present results from applying the \HAPD{} algorithm to various types of numbers, demonstrating its effectiveness in identifying cubic irrationals.

\begin{table}[h]
\centering
\caption{Results of the \HAPD{} Algorithm for Different Number Types}
\label{tab:hapd_results}
\begin{tabular}{|l|l|l|l|l|}
\hline
\textbf{Number} & \textbf{Type} & \textbf{Behavior} & \textbf{Preperiod} & \textbf{Period} \\
\hline
$\sqrt{2}$ & Quadratic Irrational & Non-periodic & - & - \\
$\sqrt{3}$ & Quadratic Irrational & Non-periodic & - & - \\
$\frac{1+\sqrt{5}}{2}$ & Quadratic Irrational & Non-periodic & - & - \\
\hline
$2^{1/3}$ & Cubic Irrational & Periodic & 1 & 2 \\
$3^{1/3}$ & Cubic Irrational & Periodic & 1 & 3 \\
$1+2^{1/3}$ & Cubic Irrational & Periodic & 0 & 4 \\
\hline
$\pi$ & Transcendental & Non-periodic & - & - \\
$e$ & Transcendental & Non-periodic & - & - \\
\hline
$\frac{3}{2}$ & Rational & Terminates & - & - \\
$\frac{22}{7}$ & Rational & Terminates & - & - \\
\hline
\end{tabular}
\end{table}

\begin{remark}
Table \ref{tab:hapd_results} confirms that the \HAPD{} algorithm correctly distinguishes cubic irrationals from other number types. Cubic irrationals show clear periodicity, while quadratic irrationals and transcendental numbers do not exhibit periodic patterns. Rational numbers cause the algorithm to terminate early, as expected.
\end{remark}

\begin{example}[Cube Root of 2 Analysis]
For $\alpha = 2^{1/3}$, the \HAPD{} algorithm produces the following sequence:
\begin{enumerate}
    \item Initial triple: $(1.2599, 1.5874, 1.0000)$
    \item Iteration 1: $(a_1, a_2) = (1, 1)$, new triple: $(0.2599, 0.5874, 0.1527)$
    \item Iteration 2: $(a_1, a_2) = (1, 3)$, new triple: $(0.1072, 0.1293, -0.3426)$
    \item Iteration 3: $(a_1, a_2) = (-1, -1)$, new triple: $(-0.2354, -0.2133, -0.7914)$
    \item Iteration 4: $(a_1, a_2) = (0, 0)$, new triple: $(-0.2354, -0.2133, -0.7914)$
\end{enumerate}

Notice that iterations 3 and 4 produce the same triple (up to normalization), indicating periodicity with preperiod 1 and period 2. The pattern of pairs $(a_1, a_2)$ is: $(1, 1), (1, 3), (-1, -1), (0, 0), (0, 0), \ldots$
\end{example}

\begin{proposition}[False Periodic Detection in Numerical Implementation]
When implementing the \HAPD{} algorithm with floating-point arithmetic, non-cubic irrationals may appear to have periodic sequences due to:
\begin{enumerate}
    \item Limited precision causing different projective points to appear equivalent
    \item Numerical error accumulation over many iterations
    \item Inability to represent exact algebraic relations in floating-point
\end{enumerate}
\end{proposition}

\begin{proof}
In a floating-point implementation, numbers are represented with finite precision. For a quadratic irrational like $\sqrt{2}$, the relation $(\sqrt{2})^2 = 2$ cannot be represented exactly, introducing small errors.

Over many iterations, these errors can accumulate, potentially causing the algorithm to detect false periodicity. This does not contradict our theoretical results, which assume exact arithmetic. Rather, it highlights the gap between theoretical mathematics and computational implementations.

To mitigate this issue, higher precision and more sophisticated comparison methods can be used, but the fundamental limitation of floating-point arithmetic in representing exact algebraic relations remains.
\end{proof}

\subsection{Matrix Approach Implementation}

We now implement the matrix-based approach as an alternative method for detecting cubic irrationals.

\begin{algorithm}
\caption{Matrix-Based Cubic Irrational Detection}
\label{alg:matrix_detection}
\begin{algorithmic}[1]
\Procedure{DetectCubicIrrational}{$\alpha$, tolerance}
    \State Compute approximate minimal polynomial $p(x) = x^3 + ax^2 + bx + c$
    \State Create companion matrix $C = \begin{pmatrix} 0 & 0 & -c \\ 1 & 0 & -b \\ 0 & 1 & -a \end{pmatrix}$
    
    \State Initialize $I$ as the $3 \times 3$ identity matrix
    \State $C^1 \gets C$
    \State $C^2 \gets C \cdot C$
    \State $C^3 \gets C^2 \cdot C$
    
    \State $\text{traces} \gets [\text{tr}(I), \text{tr}(C^1), \text{tr}(C^2), \text{tr}(C^3)]$
    
    \State $\text{powers} \gets [3, \alpha, \alpha^2, \alpha^3]$
    
    \State $\text{consistent} \gets \text{true}$
    \For{$k \gets 1$ to $3$}
        \State Compute expected power sum $s_k$ using recurrence relation
        \If{$|\text{traces}[k] - s_k| > \text{tolerance}$}
            \State $\text{consistent} \gets \text{false}$
        \EndIf
    \EndFor
    
    \If{$\text{consistent}$}
        \State \Return "Likely cubic irrational with minimal polynomial $p(x)$"
    \Else
        \State \Return "Not a cubic irrational"
    \EndIf
\EndProcedure
\end{algorithmic}
\end{algorithm}

\begin{example}[Matrix Method for Cube Root of 2]
For $\alpha = 2^{1/3}$ with minimal polynomial $p(x) = x^3 - 2$:
\begin{enumerate}
    \item Companion matrix: $C = \begin{pmatrix} 0 & 0 & 2 \\ 1 & 0 & 0 \\ 0 & 1 & 0 \end{pmatrix}$
    \item Traces: $\tr(I) = 3$, $\tr(C) = 0$, $\tr(C^2) = 0$, $\tr(C^3) = 6$
    \item Power sums: $s_0 = 3$, $s_1 = \alpha + \beta + \gamma = 0$, $s_2 = \alpha^2 + \beta^2 + \gamma^2 = 0$, $s_3 = \alpha^3 + \beta^3 + \gamma^3 = 6$
\end{enumerate}

The traces match the expected power sums, confirming that $\alpha$ is a cubic irrational.
\end{example}

\begin{proposition}[Comparison of Methods]
The matrix-based detection method:
\begin{enumerate}
    \item Requires fewer iterations than the \HAPD{} algorithm
    \item Needs an initial guess of the minimal polynomial
    \item Is less affected by floating-point precision issues in trace calculations
    \item Provides direct verification of the minimal polynomial
\end{enumerate}
\end{proposition}

\begin{proof}
The matrix method requires only a fixed number of trace calculations (typically 3-4) once a candidate minimal polynomial is identified. This is more efficient than the $O(M^3)$ iterations needed by the \HAPD{} algorithm to detect periodicity.

However, the matrix method requires first finding a candidate minimal polynomial, which itself can be computationally challenging without prior knowledge. The \HAPD{} algorithm works directly with the real number value.

Trace calculations involve straightforward matrix operations that are generally more stable numerically than the projective transformations and equivalence checks in the \HAPD{} algorithm.

The matrix method directly verifies the coefficients of the minimal polynomial, providing explicit algebraic information about the cubic irrational.
\end{proof}

\subsection{Combined Approach and Practical Algorithm}

Based on our findings, we propose a combined approach that leverages the strengths of both methods for practical detection of cubic irrationals.

\begin{algorithm}
\caption{Combined Cubic Irrational Detection}
\label{alg:combined_detection}
\begin{algorithmic}[1]
\Procedure{DetectCubicIrrational}{$\alpha$, max\_iterations, tolerance}
    \State Run HAPD algorithm for initial\_iterations (e.g., 20)
    \If{HAPD terminates early}
        \State \Return "Rational number"
    \EndIf
    
    \If{HAPD detects clear periodicity}
        \State Use periodic pattern to reconstruct minimal polynomial
        \State Verify with matrix method
        \State \Return "Confirmed cubic irrational"
    \EndIf
    
    \State Apply PSLQ or LLL algorithm to find minimal polynomial
    \If{degree of minimal polynomial $= 3$}
        \State Verify with matrix method
        \State \Return "Likely cubic irrational"
    \ElsIf{degree of minimal polynomial $= 2$}
        \State \Return "Quadratic irrational"
    \ElsIf{degree of minimal polynomial $= 1$}
        \State \Return "Rational number"
    \Else
        \State \Return "Higher degree irrational or transcendental"
    \EndIf
\EndProcedure
\end{algorithmic}
\end{algorithm}

\begin{remark}
This combined approach balances efficiency with reliability. The \HAPD{} algorithm is used for initial screening, potentially identifying rational numbers quickly and providing evidence of periodicity for cubic irrationals. For cases where periodicity is not immediately clear, we fall back to more traditional methods like PSLQ or LLL to find a candidate minimal polynomial, then verify using the matrix method.
\end{remark}

\subsection{Implementation Guidelines and Best Practices}

Based on our experimental results and theoretical analysis, we offer the following concrete implementation guidelines for the reliable detection of cubic irrationals:

\begin{enumerate}
    \item \textbf{Precision Requirements:} 
    \begin{itemize}
        \item For cubic irrationals with coefficients $|a|, |b|, |c| \leq 100$: Use at least 128-bit (quad) precision.
        \item For coefficients $|a|, |b|, |c| \leq 1000$: Use at least 256-bit precision.
        \item General rule: Use $\approx 8 \cdot \log_2(M)$ bits of precision where $M = \max(|a|, |b|, |c|)$.
    \end{itemize}
    
    \item \textbf{Tolerance Settings:}
    \begin{itemize}
        \item For the \HAPD{} algorithm: Set projective comparison tolerance to $\varepsilon \approx 2^{-p/2}$ where $p$ is bits of precision.
        \item For the matrix method: Set trace comparison tolerance to $\varepsilon \approx 2^{-p/2+\log_2(n)}$ where $n$ is matrix dimension.
    \end{itemize}
    
    \item \textbf{Performance Optimizations:}
    \begin{itemize}
        \item Cache matrix powers in the matrix verification method rather than recomputing.
        \item Normalize projective triples only when comparing, not after each iteration.
        \item Use sparse matrix operations for companion matrices, which have a specific pattern.
    \end{itemize}
    
    \item \textbf{Periodicity Detection:}
    \begin{itemize}
        \item Store normalized triples in a hash table for faster lookups.
        \item Consider a sliding window approach for detecting periods in longer sequences.
        \item Verify potential periods with multiple consecutive matches to avoid false positives.
    \end{itemize}
\end{enumerate}

The following code snippet illustrates the core of the projective comparison function in Python with mpmath for arbitrary precision:

\begin{verbatim}
def projectively_equivalent(triple1, triple2, tolerance=1e-12):
    # Normalize triples to unit vectors
    norm1 = mp.sqrt(sum(x*x for x in triple1))
    norm2 = mp.sqrt(sum(x*x for x in triple2))
    
    unit1 = [x/norm1 for x in triple1]
    unit2 = [x/norm2 for x in triple2]
    
    # Compute dot product
    dot_product = sum(a*b for a, b in zip(unit1, unit2))
    
    # Check if dot product is close to 1 (same direction)
    # or -1 (opposite direction)
    return abs(abs(dot_product) - 1) < tolerance
\end{verbatim}

These guidelines balance theoretical rigor with practical implementation concerns, providing a framework for reliable cubic irrational detection in real-world applications.

\subsection{Implementation Challenges and Solutions}

We conclude this section by discussing practical challenges in implementing our methods and proposing solutions.

\begin{enumerate}
    \item \textbf{Precision Requirements:} For reliable detection of cubic irrationals with large coefficients, extended precision arithmetic is necessary. Libraries like MPFR for C/C++ or mpmath for Python provide arbitrary precision floating-point arithmetic.
    
    \item \textbf{Periodicity Detection:} Detecting periodicity in the presence of numerical errors requires careful design of the comparison function. Normalizing triples and using an appropriate tolerance based on the precision helps mitigate false positives and negatives.
    
    \item \textbf{Minimal Polynomial Finding:} For the matrix approach, finding a candidate minimal polynomial can be challenging. The PSLQ algorithm or lattice reduction methods (LLL) can be used, but require careful selection of basis size and precision.
    
    \item \textbf{Efficiency Considerations:} For large-scale applications, optimizing the computation of matrix powers and implementing early termination conditions can significantly improve performance.
    
    \item \textbf{Edge Cases:} Special care is needed for numbers very close to rational values or with minimal polynomials having very large coefficients, as these can require exceptional precision to distinguish accurately.
\end{enumerate}

\begin{remark}
Despite these challenges, our numerical experiments confirm that both the \HAPD{} algorithm and the matrix approach can be successfully implemented to detect cubic irrationals with high reliability. The combined approach offers a practical solution that balances theoretical rigor with computational efficiency.
\end{remark}

This comprehensive validation demonstrates that our solution to Hermite's problem is not merely a theoretical construct but a practically implementable method for detecting cubic irrationals.
