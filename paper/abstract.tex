\begin{abstract}
This paper extends Karpenkov's work on Dirichlet groups and projective algorithms \cite{Karpenkov2022} to present a comprehensive approach to Hermite's problem through the development of the Hermite-like Algorithm with Projective Dual action (\HAPD). After rigorously establishing that cubic irrationals cannot have periodic continued fraction expansions—a result that explains why direct analogies to the quadratic case fail—I enhance the projective space approach to develop an algorithm that fully characterizes cubic irrationals. I demonstrate that the \HAPD{} algorithm produces an eventually periodic sequence if and only if its input is a cubic irrational, thereby providing a perfect generalization of Lagrange's theorem on continued fractions.

The paper provides multiple independent analyses of this characterization: (1) a direct algorithmic approach analyzing the projective transformation properties of the \HAPD{} algorithm; (2) a matrix-theoretical approach involving companion matrices and their trace properties; and (3) a Galois-theoretic explanation that situates these results within the broader context of algebraic number theory. I address all potential edge cases, including cubic irrationals with different Galois group structures and the behavior of the algorithm under numerical approximation.

Extensive numerical validation confirms that the \HAPD{} algorithm correctly distinguishes cubic irrationals from other number types with high precision. While Karpenkov's sin²-algorithm offered a solution for the totally-real cubic case, my approach extends to all cubic irrationals, contributing to the resolution of a 170-year-old question posed by Hermite.
\end{abstract}
