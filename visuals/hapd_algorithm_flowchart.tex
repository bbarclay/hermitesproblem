% HAPD Algorithm Flowchart
% Professional redesign with proper spacing and clear layout

\begin{figure}[ht]
\centering
\begin{tikzpicture}[
    scale=0.8, % Slightly reduce scale to fit better
    % Clean, modern styles with plenty of space
    block/.style={rectangle, draw=blue!50, rounded corners, fill=blue!5, 
                 minimum width=7cm, minimum height=1.6cm, % Increased height 
                 text centered, font=\sffamily, align=center},
    io/.style={rectangle, draw=green!50, rounded corners, fill=green!5, 
              minimum width=7cm, minimum height=1.6cm, % Increased height
              text centered, font=\sffamily, align=center},
    decision/.style={diamond, draw=orange!50, fill=orange!5, 
                    minimum width=4cm, minimum height=4cm, % Larger diamond
                    text centered, align=center, font=\sffamily,
                    inner sep=3pt}, % More padding
    arrow/.style={thick, ->, >=stealth},
    note/.style={font=\sffamily\small, align=center, text width=3cm}
]
    % Clean background for the entire diagram - make taller
    \fill[rounded corners=15pt, gray!3, draw=gray!10] 
        (-8,8) rectangle (8,-9.5);

    % Title - properly positioned higher up
    \node[font=\bfseries\sffamily\large, align=center] at (0,7) 
        {HAPD Algorithm: Periodicity Detection};

    % Input node - more spacing between nodes
    \node[io] (input) at (0,5.0) {Input: Cubic irrational $\alpha$};

    % Initialization - clear description - increased vertical spacing
    \node[block] (init) at (0,2.8) 
        {Initialize projective coordinates\\
         $(v_1, v_2, v_3) = (\alpha, \alpha^2, 1)$};

    % Iteration with clear formula - increased vertical spacing
    \node[block] (iterate) at (0,0.4) 
        {Compute floor quotients:\\
         $a_1 = \lfloor v_1/v_3 \rfloor$, 
         $a_2 = \lfloor v_2/v_3 \rfloor$};

    % Update step - clear description - increased vertical spacing
    \node[block] (update) at (0,-2.0) 
        {Update triple:\\
         $(v_1, v_2, v_3) \leftarrow$ (remainders after division)};

    % Decision diamond - much larger with centered text and more space
    \node[decision] (check) at (0,-5.0) 
        {Is the current\\triple projectively\\equivalent to a\\previous one?};

    % Output - clear result - moved down for better spacing
    \node[io] (output) at (0,-8.0) 
        {Output: Period detected,\\cubic irrational confirmed};

    % Connect with clean arrows - adjusted for new positions
    \draw[arrow] (input) -- (init);
    \draw[arrow] (init) -- (iterate);
    \draw[arrow] (iterate) -- (update);
    \draw[arrow] (update) -- (check);
    \draw[arrow] (check) -- node[right, note, xshift=0.3cm] {Yes} (output);

    % Loop back - wider path to avoid crowding
    \draw[arrow] (check) -- node[note, below, yshift=-0.3cm] {No} ++(6.0,0) -- ++(0,5.0) -- ++(-6.0,0) -- (iterate);

\end{tikzpicture}
\caption{Flowchart of the HAPD algorithm for detecting periodicity in cubic irrationals. The algorithm transforms points in projective space and checks for repetition, which confirms that the input is a cubic irrational.}
\label{fig:hapd_flowchart}
\end{figure} 